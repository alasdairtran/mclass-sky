%%
%% Template abstract.tex
%%

\chapter*{Abstract}
\label{cha:abstract}
\addcontentsline{toc}{chapter}{Abstract}

Recent sky surveys such as the Sloan Digital Sky Survey and the VST ATLAS Survey have given us a large
amount of data to work with. Spectroscopic labelling, however, is quite expensive, so we only
want to manually label an object if doing so allows us to gain new information. This thesis
explores pool-based active learning and the novel application of prominent active learning heuristics to
the domain of photometric classification.

We begin by applying standard supervised machine learning techniques to two astronomical datasets.
The best-performing classifiers with reliable probability estimates, logistic regression and support
vector machines, are then used to conduct the active learning experiment. Our key original
contribution is the application of Thompson sampling, a Bayesian solution to the exploration vs
exploitation problem, to the selection of six active learning heuristics. To address the problem of
class imbalance, we derive an extension of the posterior balanced accuracy to the
multi-class setting. This is used to evaluate the performance of our algorithms.

The results are very promising. Even under simplistic assumptions like a normally distributed
reward, Thompson sampling manages to automatically identify the optimal heuristic after only 50
examples. In particular, the margin minimisation technique is a clear winner, outperforming random
sampling by as much as 9\% in the VST ATLAS dataset after 300 examples. Being very
computationally efficient, we recommend that astronomers use the margin heuristic with logistic
regression to decide which objects to label next in future sky surveys. To help them with this task,
we have created an open-source and extendable Python package that allows others to easily apply
active learning routines to any dataset and make quick visualisation of photometric data.


%%% Local Variables: 
%%% mode: latex
%%% TeX-master: "thesis"
%%% End: 

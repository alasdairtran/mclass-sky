
%%
%% Template chap2.tex
%%

\chapter{Photometric Classification}
\label{cha:ml}

The two most common types of celestial objects are stars and galaxies. There are also some other
interesting objects such as quasars and white dwarfs. Quasars are thought to be supermassive black 
holes surrounded by an accretion disc. They are very luminous and, unlike galaxies, appear as
single-source objects like stars.
White dwarfs are low to intermediate mass stars that are in their final evolutionary stage.
They are very dense and have a faint luminosity.

One way to classify objects into 
these various groups is to manually inspect their spectra.
There have even been attempts to make the process
more automatic. For example, \citeN{hala14} achieved a 95\% accuracy rate by training
a convolutional neural network on one-dimensional spectra to classify objects
into stars, quasars, and galaxies. Even so, it is currently not possible to obtain
a spectrum of every object, especially faint ones. This means only a small number of objects
(e.g. less than 0.05\% in the SDSS dataset) can be classified this way. For the rest,
we only have photometric measurements.

Fortunately, the field of machine learning came about to solve this kind of problem.
In the most basic set-up, we have a set of objects, each with a vector of photometric measurements
$\bm{x} \in \X$. A subset of them forms the labelled set. Each object in this set
has been spectroscopically classified into some class $y \in \Y$.
We call $\X$ the feature space and $\Y$ the label space.
During the training phase, we feed labelled examples to a classifier,
which will attempt to form a hypothesis $h: \X \rightarrow \Y$. These labelled examples form
the training set and they are of course not perfect, so
the goal in machine learning is to capture as much of the underlying trend in the data as possible,
while avoiding fitting the random noise. Once trained, the classifier can then
be used to predict the classes of unlabelled objects.

\section{Classifiers}
\label{sec:machine}
Three family of classifiers are used in our experiments. They are random forests,
logistic regression, and support vector machines. Below we give a quick overview of how
each of them works. We do not implement these classifiers ourselves. Rather we use
scikit-learn \cite{pedregosa11}, the most well-known machine learning package
in Python. 

\subsection{Random Forest}

To understand the motivation behind random forests, we first need to know how to construct
individual decision trees. 
Building each of these trees is like playing the a game of Twenty Questions. We start with the
whole training set and at each step, we slice the feature space along the axis of one feature.
After many steps, we end up with a set of hyper-dimensional cuboids which form a partition of
the feature space. The algorithm stops when each of these cuboids
contains data from only one class. There are many
criteria that we can use to decide on which feature and where we slice along the axis.
In this thesis, we use the Gini impurity which intuitively measures the potential misclassification
rate. In particular, let $k$ be the number of classes and $q_D(i)$ be the frequency of objects
belonging to class $i$ in set $D$. Then the Gini impurity of $D$ is the 
probability that a randomly selected object from $D$ is misclassified if it were labelled
according to its frequency in $D$:
	\begin{IEEEeqnarray*}{lCl}
		\iota_G(D) &=& \sum_{i=1}^{k} q_D(i) (1 - q_D(i)) \\
		           &=& \sum_{i=1}^{k} q_D(i)  - \sum_{i=1}^{k} q_D(i)^2 \\
		           &=& 1 - \sum_{i=1}^{k} q_D(i)^2
	\end{IEEEeqnarray*}
When we partition $D$ into subsets $\{D_1, D_2, ..., D_d\}$, the Gini impurity of $D$ is now
the sum of the individual Gini impurities, weighted by the size of the subsets:
	\begin{IEEEeqnarray*}{lCl}
		\iota_G(D) &=& \sum_{i=1}^{d} \frac{|D_i|}{|D|} \iota_G(D_i)
	\end{IEEEeqnarray*}
Observe that if a subset contains objects from only one class, then its Gini impurity would be
zero. This gives us the following splitting criterion: at each step, slice the feature space
along the axis that will result in the greatest drop in the Gini impurity.

One problem with decision trees is that they tend to overfit the data and thus do not generalise
well. To solve this, \citeN{breiman01} proposes that we build many decision trees,
thus creating a random forest.
The random forest makes its prediction by simply counting up the
predictions of all the individual trees and then taking the most popular choice. By averaging
the predictions, we avoid the problem of overfitting. Furthermore, for each tree,
we only give it a small bootstrap sample and at each split, we only consider a small number
of features. This bootstrapping and random subspace selection have been shown empirically
to improve the accuracy rate \cite{breiman96, ho98, louppe12}.
Another nice
feature of random forests is that they are extremely fast and hence scale well with large
datasets. Although they do not provide class probability estimates, we can use proportions
of the votes as a proxy for the likelihood that an object is in each class.
However, as we shall see in Appendix \ref{sec:forest_prob}, these probabilities
are not very stable.

\subsection{Logistic Regression}
If we want to have better probability estimates, then an alternative approach is to
use logistic regression.
Developed by \citeN{cox58}, the algorithm tries to directly model the probability of 
being in a class. Let $\bm{x}$ be the feature vector and $\bm{\beta}$ be the vector of 
coefficients. The linear predictor $\eta$ is defined as
	\begin{IEEEeqnarray*}{lCl}
		\eta(\bm{x}) &=& \bm{\beta}^T \bm{x}
	\end{IEEEeqnarray*}	
Since probabilities must lie between 0 and 1, we want our predictor to have the same range.
This can be achieved by wrapping $\eta$ around the logistic function:
	\begin{IEEEeqnarray*}{lCl}
		\sigma(\eta(\bm{x})) &=& \frac{1}{1 + e^{-\eta(\bm{x})}}
	\end{IEEEeqnarray*}
We can now interpret $\sigma(\eta(\bm{x}))$ as the probability that an object with feature
vector $\bm{x}$ belongs to the positive class. The goal of the algorithm is then to use the
training data to estimate the coefficient vector $\bm{\beta}$.

There are a few ways to extend this model to the multi-class setting. One option is
the multinomial logistic regression, where we would need to jointly solve a set of $(k-1)$ 
binary regressions if we have $k$ classes. In practice, when running the multinomial option
in scikit-learn, the probability estimates are as unreliable as those of random forests.
The cause of this is unknown, but it is more likely due to flaws in the scikit-learn
implementation than in the actual theory.\footnote{See Figure ~ in
	Appendix ~ for a plot comparing the learning curves of multinomial with one-vs-rest.}
A more empirically stable alternative is to use the one-vs-rest strategy, where we run $k$ 
independent binary regressions. In particular, for class $i$, we pretend that the dataset
contains only objects from class $i$ and class `not $i$'. We then train the binary logistic model
on this simplified dataset and we do this $k$ times, one for each class. At the end, we end up
with $k$ probabilities. These can be interpreted like usual after normalisation.


\subsection{Support Vector Machines}

Support vector machines (SVM), first introduced by \citeN{boser92}, are another popular
family of algorithms. They have been used in astronomy, for example by \citeN{elting08},
to find non-linear decision boundaries in the colour space of the SDSS dataset. The idea
here is to find a decision boundary that can maximise the distance between the boundary and the
closest data points. This can be done by solving a Lagrangian function. SVMs, in their original formulation, are inherently binary classifiers.
However we can still use the one-vs-rest strategy to extend it to the multi-class setting.
There has even been an attempt to derive an inherently multi-class SVM \cite{crammer02}.

\subsection{Learning Complex Hypotheses}

Both SVMs and logistic regression are linear classifiers. When dealing with real-world data
like those in astronomy, we should not expect to be able to separate classes with a
hyper-dimensional plane.
If we want them to learn more complex hypotheses, one option
is do an explicit polynomial transformation of the original photometric measurements. When we give
the classifier the transformed features, it would still find a linear boundary in the transformed
space. However, the boundary would mostly be non-linear in the original space. A second
option is to use the kernel trick which does an implicit map into a high-dimensional feature space.
For example, a popular kernel that is often used with SVMs is the radial basis function (RBF),
which actually maps the inputs into an infinite-dimensional space.

In random forests, we do not have to worry about any transformation. Although in each round,
we slice the data along only one axis, there is no limit on how many slices we can take and
how small the resulting cuboids can be. This allows us to learn arbitrarily complex
hypotheses.

\section{Overview of Active Learning}

We now turn our attention to the construction of the training set.
Getting spectroscopic labels is expensive. Until now, when astronomers pick which objects to obtain
spectra from, not much attention is paid to whether the labelling information can improve
the classifier performance. To see why this might be a problem, imagine that there is a group of objects
with very similar photometric measurements. We can obtain spectra from all of them and conclude,
for example,
that they are all stars. However, a smarter way is to get only one spectrum from this group
for labelling and 
let the classifier generalise to other similar objects. Keeping the size of the training set
as small as possible while not sacrificing the classifier accuracy is the goal of active
learning.

We will focus on pool-based active learning, the most relevant type of active learning
for astronomy. In this setting, we keep track of two sets. The labelled
set $\Labelled \subset \X \times \Y$ contains all examples that have been labelled
by an expert so far. All the remaining unlabelled examples form the unlabelled set $\Unlabelled \subset \X$.
The question now is how to select the next training example from $\Unlabelled$. In practical terms, 
where should we next point the telescope to, in order to obtain a spectrum? To answer this question,
we need a rule $r(\bm{x}; h)$ that can assign a score to each object, based solely on their
photometric
features and the current hypothesis. From now on we will simply write $r(\bm{x})$ to avoid symbol overloading. This score should reflect the amount of new information we would gain if we were to label the object. 
We will discuss various heuristics in the next section. Once we have computed $r(\bm{x})$ for
all candidates,
we can then pick the example with the highest score and obtain its spectrum. The object's
features
and its label are then added to the training set and the classifier is retrained to obtain a new $h$.

In practice, the unlabelled pool can be arbitrarily large. For example, there are 800 million 
unlabelled objects in the SDSS. Thus if we only have a limited computing power, at each round,
we might only be able to select a small sample of size $t$ where the score of each can
be calculated.
A formal description of the active learning routine is given as follows. Note that
for some active learning rules, we might need to substitute $\argmax$ with $\argmin$ in
line 4.

\algblock[Name]{Start}{End}

\algblockdefx[Forall]{Foreach}{Endforeach}%
			[1]{\textbf{for each} #1 \textbf{do}}%
			{\textbf{end for}}

\begin{algorithm}[h]
	\caption{The general pool-based active learning algorithm}
	\label{alg:active}
	\begin{algorithmic}[1]
		\Procedure {ActiveLearner}{$\Unlabelled$, $\Labelled$, $h$, $r$ $n$, $t$}
			\While {$|\Labelled| < n$}
				\State $E$ $\leftarrow$ random sample of size $t$ from $\Unlabelled$
				\State $\bm{x}_* \leftarrow \argmax_{\bm{x} \in E} r(\bm{x})$
				\State $y_* \leftarrow$ ask the expert to label $\bm{x}_*$
				\State $\Labelled \leftarrow \Labelled  \cup (\bm{x}_*, y_*)$
				\State $\Unlabelled \leftarrow \Unlabelled \setminus \bm{x}_*$
				\State $h_\Labelled(\bm{x}) \leftarrow$ retrain the classifier
			\EndWhile
			\EndProcedure
	\end{algorithmic}
\end{algorithm}


\section{Active Learning Heuristics}
\label{sec:heuristics}

There are many ways to rank unlabelled objects. The four prominent families
of heuristics are uncertainty sampling, query by bagging, loss function minimisation, and
and classifier certainty \cite{schein07}. All of these heuristics require the
class probabilities estimated by the current hypothesis. 
We now discuss each of them in turn, starting with the least computationally expensive
one.

\subsection{Uncertainty Sampling}

\citeN{lewis94} introduce the idea of uncertainty sampling, where we select the example
whose class membership the classifier is least certain about.
These tend to be points that are near the decision boundary of the classifier. To quantify
the uncertainty, \citeN{schein07} propose two techniques, entropy maximisation and
margin minimisation.

The Shannon entropy measures the amount of information contained in some object $\bm{x}$
and is defined as
	\begin{IEEEeqnarray*}{lCl}
		r_S(\bm{x}) &=& -\sum_{c \in \Y} \Prob{y(\bm{x}) = c} \log \big[\Prob{y(\bm{x}) = c} \big]
	\end{IEEEeqnarray*}
Intuitively, the closer class probabilities of an object
are to random guessing, the higher its entropy will be. This gives us the heuristic
of picking the candidate with the highest Shannon entropy.

In fact, if we care about how close the class probabilities are to random guessing, 
there is an even simpler measure. 
Let $c^{(1)}$ and $c^{(2)}$ be the two most likely classes for some object $\bm{x}$.
We define the margin as the difference between these two values:
	\begin{IEEEeqnarray*}{lCl}
		r_M(\bm{x}) &=& \Big|  \Prob{y(\bm{x}) = c^{(1)}} -  \Prob{y(\bm{x}) = c^{(2)}} \Big|
	\end{IEEEeqnarray*}
Since the class probabilities must add up to 1, the smaller the margin is,
the more uncertain we are. Thus another heuristic is
to pick the candidate with the smallest margin.


\subsection{Query by Bagging}

Instead of focussing the uncertainty of individual predictions, we could look at
the disagreement among a group of classifiers. In query by bagging (QBB), before we
can assign a score to each candidate, we need to train a committee of classifiers, each
with a hypothesis that is as different from the others as possible but that is still consistent
with the training data \cite{melville04}. This will hopefully reduce the noise when
we take the average of the member predictions. In order to have this variation,
each committee member is only given a subset of the labelled examples. Since there
might be enough training data (for example, in our experiments, we have 11 members but only a
maximum of 300 labelled points), we need to select samples with replacement (hence the term bagging).

Once we've constructed the committee, we pick the candidate whose class membership the committee 
disagrees the most about about. To quantify the disagreement, \citeN{melville04} extend the
margin approach from uncertainty sampling. This involves averaging out the class probabilities estimated
by the committee members and then calculating the margin. Since the bagging technique
has been shown to improve the stability of the predictions \cite{breiman96}, we should
expect this method to be no worse the the simple margin approach. The cost, however, is that
it now takes $B$ times longer to calculate the scores, where $B$ is the size of the 
committee.

In addition to the margin, \citeN{mccallum98} offer an alternative disagreement measure which
involves the Kullback–Leibler (KL) divergence. Suppose we have two discrete
probability distributions $P$ and $Q$. Then the KL divergence of $P$ from $Q$ is defined as
	\begin{IEEEeqnarray*}{lCl}
		D_{\mathrm{KL}}(P\|Q) = \sum_i P(i) \, \ln\frac{P(i)}{Q(i)}
	\end{IEEEeqnarray*}
This KL divergence measures the amount of information lost when $Q$ is used to approximate $P$. Intuitively, the larger the KL divergence is, the more disagreement there is between $Q$ and $P$. In the active learning context, $Q$ is the average prediction probability of the committee, while $P$ is the prediction of a particular committee member.
This gives us the heuristic of picking the candidate with the largest expected KL divergence from the average:
	\begin{IEEEeqnarray*}{lCl}
		r_{QBB~KL}(\bm{x}) = \dfrac{1}{B} \sum_{b=1}^B D_{\mathrm{KL}}(P_b\|Q)
	\end{IEEEeqnarray*}




\subsection{Variance Minimisation}
We can in fact try to minimise a loss function explicitly. For example, the expected
squared loss can be decomposed into three terms:
	\begin{IEEEeqnarray*}{lCl}
		\E{\text{Squared Loss}} &=& \text{Bias}^2 + \text{Variance} + \text{Noise}
	\end{IEEEeqnarray*}
The noise is intrinsic in the data and represents the expected loss under the optimal hypothesis.
Thus there is nothing we can do about the noise. The squared bias can be minimised \cite{cohn97},
but we do not explore it here. Instead, let us focus on minimising the variance of
the unlabelled pool since
the term will vanish as the training set size approaches infinity. Using Taylor series approximation,
\citeN{schein07} estimate the variance in multinomial logistic regression as
	\begin{IEEEeqnarray*}{lCl}
		\text{V}_{\Labelled} &=& tr(AF^{-1})
	\end{IEEEeqnarray*}
We subscript the variance with the labelled training set since as we change the training set,
the hypothesis and then the probability estimates change.
Here $F$ is the Fisher information matrix and
	\begin{IEEEeqnarray*}{lCl}
		A &=& \sum_n \sum_c \mathbf{g}_n(c) ~ \mathbf{g}_n(c)^T
	\end{IEEEeqnarray*}
where $\mathbf{g}_n(c)$ is the gradient vector. The allows us to create the following
active learning heuristic: pick the candidate that is expected the provide the greatest
decrease in the variance if were part of the training set:
	\begin{IEEEeqnarray*}{lCl}
		r_V(\bm{x}) &=& -\sum_i^k \Prob{y(\bm{x}) = i} V_{\Labelled \cup \bm{x}}
	\end{IEEEeqnarray*}
This is quite an expensive computation. To calculate the score of only
one candidate, we first need to give it each the possible labels, add it in the training set to get
a updated hypothesis, and compute the new variance. The expectation is over the 
class probability distribution of the current hypothesis.

Note also that this derivation is specific to
the multinomial logistic regression. If we use the one-vs-rest strategy with binary logistic
regression or another entirely different classifier like SVMs in our experiments,
the same approximation might not hold and we should not expect to get good results.
We leave the variance estimation of other learning algorithms for future work.



\subsection{Classifier Certainty}
Finally, instead of minimising the variance of the unlabelled pool, \citeN{mackay91} proposes
minimising the entropy of the classifier's predictions on $\Unlabelled$, which is calculated as
	\begin{IEEEeqnarray*}{lCl}
		CC_{\Labelled} &=& - \sum_{\bm{u} \in \Unlabelled} \sum_{c \in \Y} \Prob{y(\bm{u}) = c} \log \big[\Prob{y(\bm{u}) = c} \big]
	\end{IEEEeqnarray*}
Although this sounds
similar to one of the uncertainty sampling heuristics, here instead of picking the 
most uncertain candidate, we pick the candidate that is expected to increase the classifier's
prediction certainty by the the most amount:
	\begin{IEEEeqnarray*}{lCl}
		r_{CC}(\bm{x}) &=& -\sum_{c\in \Y} \Prob{y(\bm{x}) = c} CC_{\Labelled \cup \bm{x}}
	\end{IEEEeqnarray*}
Like the variance estimation, to get the score of just one candidate, we need to retrain the
classifier $k$ times, where $k$ is the number of classes. Thus in practice, both the variance
estimation and the classifier certainty heuristics might be too computationally expensive to run.


\section{Multi-arm Bandit}
Let us call the six heuristics we have just described entropy, margin, QBB margin, QBB KL,
pool variance, and pool entropy, respectively. Out of these six, how
do we know which  heuristic to the best, anyway?
There have been some attempts in the literature to do a theoretical analysis of them.
However proofs are quite scarce, and when there is one available, they normally only
work under simplifying assumptions. For example, \shortciteN{freund97} showed that 
query by committee heuristics (where we sample without replacement) guarantee
an exponential decrease in the prediction error with the training size, but only in
the noiseless case.
Thus whether many of these are guaranteed to beat random sampling is still an open question.
We do not worry too much about theoretical analysis in this thesis. Instead we focus on
an empirical analysis in the astronomical domain.

To help us automatically choose the optimal heuristic, 
we now turn our attention to the multi-armed bandit problem in probability theory. The colourful
name originates from the situation where a gambler stands in front a slot machine with $n$ levers.
When pulled, each lever gives out a random reward according to some unknown distribution.
The goal of the game is to come up with a strategy that can maximise the gambler's
lifetime rewards with a minimum number of pulls.

The key novel contribution of this thesis is the application of this theory to the problem of
heuristic selection. Suppose we have a set of $n$ heuristics $ \R = \{r_1, ..., r_n \}$. Each heuristic
has a different ability to pick the candidate that can give the biggest gain in information
when added to the training set. An appropriate reward is then the incremental increase
in the accuracy rate. Observe that the heuristic rewards are independent of each other,
since the theories with which we use to derive the heuristics are mostly unrelated.

Let $\bm{w}$ be the reward vector where entry $\bm{w}_i$ is the reward of heuristic $r_i$.
Observe that even with the optimal heuristic, there could be error during the labelling
process that causes the accuracy rate to decrease. Conversely, a bad heuristic might be
able to pick an informative candidate due to pure luck. Thus there is always a certain level
of randomness in the reward received. These errors are probably normally distributed, so
	\begin{IEEEeqnarray*}{lCl}
		(\bm{w} \mid \bm{\nu}, \tau^2) \sim N(\bm{\nu}, \tau^2 \bm{I})
	\end{IEEEeqnarray*}
where $\bm{I}$ is the identity matrix.
To make the problem tractable, assume that we know the constant variance $\tau^2$. Assume that
the mean vector $\bm{\nu}$ follows a normal distribution
	\begin{IEEEeqnarray*}{lCl}
		\bm{\nu} \sim N(\bm{\mu}, \sigma^2 \bm{I})
	\end{IEEEeqnarray*}
Since we do not yet have any information about the performance of each heuristic,
the hyperparameters $\bm{\mu}$ and $\sigma^2$ are unknown.

One problem in multi-arm bandits is the trade-off between exploration and exploitation.
Suppose we have managed to estimate $\bm{\mu}$ and $\sigma^2$, then by always selecting
the heuristic with the highest possible $\bm{\mu}$, or the greedy heuristic,
we would be exploiting our current
knowledge. If however we select one of the other non-greedy heuristics, we would then be exploring
with the intention of improving our current estimates of $\bm{\mu}$ and $\sigma^2$. There are
many instances in which we find our previously held beliefs to be completely wrong. Thus
by always exploiting, we might miss out on the optimal heuristic. On the other hand,
if we explore too much, it might take a long time to reach the desired accuracy rate and
the strategy ends up being no different from random sampling. 

\section{Thompson Sampling}
There are two main methods in the literature that address this exploration vs exploitation 
problem. The algorithm with a strong theoretical guarantee is Upper Confidence Bound
\cite{auer02}. We will, however, focus a simpler and much older algorithm called Thompson sampling.
First introduced by \citeN{thompson33}, the algorithm solves the trade-off from
a Bayesian perspective. It has been shown to achieve results
that are comparable and sometimes even better than Upper Confidence Bound \cite{chapelle11}.

Under the heuristic selection setting, Thompson sampling works as follows.
We start a prior knowledge of $\bm{\mu}$ and $\sigma^2$. In each
round, we draw a random sample from the distribution $N(\bm{\mu}, \sigma^2 \bm{I})$
and select heuristic $r_*$ that has the highest mean reward value in the sample. We then 
observe the actual reward $w_{*}$ received and use it to update the prior accordingly.
Under the normal likelihood, the normal distribution is conjugate to itself, and it can be
shown that the posterior distribution of the mean reward $\bm{\nu}_{*}$
remains normal, but now with parameters
	\begin{IEEEeqnarray*}{lCl}
		(\bm{\nu}_{*} \mid w_{*}) \sim N \left(
		\frac{\bm{\mu}_* \bm{\tau}_* + w_{r^*} \bm{\sigma}_*}{\bm{\tau}_* + \bm{\sigma}_*},
		\frac{\bm{\sigma}_* \bm{\tau}_*}{\bm{\tau}_* + \bm{\sigma}_*}
		\right)
	\end{IEEEeqnarray*}
Below is the formal specification of the algorithm.

\begin{algorithm}[h]
	\caption{Thompson sapmling}
	\label{alg:thompson}
	\begin{algorithmic}[1]
		\Procedure {ThompsonSampling}{$\R$, $\bm{\mu}$, $\bm{\sigma}$, $\bm{\tau}$}
		\Foreach {$t \in \{1, 2, ..., n\}$}
		\Foreach {$r \in \R$}
		\State $\bm{\nu}_r' \leftarrow$ draw a sample from $N(\bm{\mu_r}, \bm{\sigma_r})$
		\Endforeach
		\State $r_* \leftarrow \argmax_{r \in \R}\bm{\nu}_r'$
		\State Observe reward $w_{*}$
		\State Update $\bm{\mu}_{*}$
		\State Update $\bm{\sigma}_{*}$
		\Endforeach
		\EndProcedure
	\end{algorithmic}
\end{algorithm}

If we combine Algorithm \ref{alg:active} and \ref{alg:thompson}, we end up with
an algorithm that can automatically select the optimal active learning heuristic
with Thompson sampling.

\begin{algorithm}[h]
	\caption{The multi-arm bandit active learning algorithm}
	\label{alg:bandit}
	\begin{algorithmic}[1]
		\Procedure {ActiveBandit}{$\Unlabelled$, $\Labelled$, $h$, $n$, $t$,
			                      $\R$, $\bm{\mu}$, $\bm{\sigma}$, $\bm{\tau}$}
		\While {$|\Labelled| < n$}
		\Foreach {$r \in \R$}
			\State $\bm{\nu}_r' \leftarrow$ draw a sample from $N(\bm{\mu_r}, \bm{\sigma_r})$
		\Endforeach
		\State $r_* \leftarrow \argmax_{r \in \R} \bm{\nu}_r'$
		\State $E$ $\leftarrow$ random sample of size $t$ from $\Unlabelled$
		\State $\bm{x}_* \leftarrow \argmax_{\bm{x} \in E} r_*(\bm{x})$
		\State $y_* \leftarrow$ ask the expert to label $\bm{x}_*$
		\State $\Labelled \leftarrow \Labelled  \cup (\bm{x}_*, y_*)$
		\State $\Unlabelled \leftarrow \Unlabelled \setminus \bm{x}_*$
		\State $h_\Labelled(\bm{x}) \leftarrow$ retrain the classifier
		\State $\delta$ $\leftarrow$ change in the balanced accuracy rate 
		\State Update $\bm{\mu}_{*}$
		\State Update $\bm{\sigma}_{*}$
		\EndWhile
		\EndProcedure
	\end{algorithmic}
\end{algorithm}

As we shall see in Chapter \ref{cha:expt2}, the reward function dynamically evolves
as the training size increases. Intuitively, the accuracy rate can never go beyond 100\%, so
we would expect the incremental change in the accuracy rate to become smaller over time.
Attempts to address this problem have been made in the literature. For example \citeN{gupta11}
introduce the Dynamic Thompson Sampling method that manages to adapt to the evolving
parameters faster than Algorithm \ref{alg:thompson}. However, we shall leave the investigation
of such method to future work.


% % % % % % % % % % % % % % % % % % % % % % % % % % % % % % % % % % % % % % % % % % % % % % % % % % 
\section{Performance Measures}
\label{sec:measures}

We end this chapter with a description of two performance measures that are used in the experiments.

\subsection{Recall}
Let $C$ be the confusion matrix of a classifier on a test set,
where entry $C_{ij}$ is the number of objects in class $i$
but have been predicted to be in class $j$.
Recall measures the classifier's ability to find all the positive examples and avoid
having false negatives:
\begin{IEEEeqnarray*}{lCl}
	\text{Recall}_i &=& \frac{C_{ii}}{\sum_j C_{ij}}
\end{IEEEeqnarray*}
We use recall to plot the the performance of a classifier on the celestial sphere. This
measure is chosen mainly due to its simple implementation.

\subsection{Posterior Balanced Accuracy Rate}
Certain astronomical objects are either rarer or more difficult to detect than others.
In the SDSS labelled set, there are 4.5 times as many galaxies as quasars. The problem
of class imbalance is even more severe in the VST-ATLAS set, with 11 times more stars than
white dwarfs. An easy fix is to undersample the dominant class when creating training and
test sets. This, of course, means that the size of these sets are limited by the size
of the minority class.

When we do not want to alter the underlying class distributions or when larger training and test
sets are desired, we need a performance measure that can correct for the class imbalance.
\shortciteN{brodersen10} showed that the posterior balanced accuracy distribution can overcome
the bias in the binary case. We now extend this idea to the multi-class setting.

Suppose we have $k$ classes. For each class $i$ between $1$ and $k$, there are $N_i$ objects
in the universe. Given a classifier, we can assign a predicted label to every object and
compare our prediction to the true label. Let $C_i$ be the number of objects in class $i$
that are correctly predicted. Then we define the accuracy rate $A_i$ of class $i$ as
	\begin{IEEEeqnarray*}{lCl}
		A_i &=& \frac{C_i}{N_i}
	\end{IEEEeqnarray*}
Initially we have no information about $C_i$ and $N_i$, so we can assume that each $A_i$ 
follows a uniform prior from 0 to 1. This is the same as a Beta distribution
with parameters $\alpha = \beta = 1$:
	\begin{IEEEeqnarray}{lCl}
		A_i &\sim& \Beta(1,1) \label{eqn:prior}
	\end{IEEEeqnarray}
After we have trained the classifier, suppose we have a test set containing $n_i$
objects in class $i$. Running the classifier on this test set is the same as conducting
$k$ binomial experiments, where, in the $i$th experiment, the sample size is
$n_i$ and the probability of success is simply the accuracy rate $A_i$. Let $c_i$ be
the number of correctly labelled objects belonging to class $i$ in the test set. Then,
conditional on the accuracy rate, $c_i$ follows a binomial distribution:
	\begin{IEEEeqnarray}{lCl}
		(c_i \mid A_i) &\sim& \Bin(n_i, A_i) \label{eqn:likelihood}
	\end{IEEEeqnarray}
In the Bayesian setting, \eqref{eqn:prior} is the prior and \eqref{eqn:likelihood}
is the likelihood. To get the posterior PDF, we simply multiply the prior with the likelihood:
	\begin{IEEEeqnarray*}{lCl}
		f_{A_i \mid \bm{c}}(a)
		&\propto& f_{A_i}(a) \times f_{c_i \mid A_i}(c_i) \\
		&\propto& a^{1-1}(1-a)^{1-1} \times a^{c_i} (1 - a)^{n_i - c_i} \\
		&=& a^{1 + c_i - 1}(1-a)^{1 + n_i - c_i - 1}
	\end{IEEEeqnarray*}
Thus, with respect to the binomial likelihood function,
the Beta distribution is conjugate to itself. The posterior accuracy rate $A_i$
also follows a Beta distribution, now with parameters
	\begin{IEEEeqnarray*}{lCl}
		(A_i \mid c_i) &\sim& \Beta(1 + c_i, 1 + n_i - c_i)
	\end{IEEEeqnarray*}
Recall that our goal is to have a balanced accuracy rate, $A$, that puts an equal
weight in each class. This can be achieved by taking the average of all the class accuracy rates:
	\begin{IEEEeqnarray*}{lCl}
		A &=& \frac{1}{k} \sum_{i=1}^k A_i \\
		&=& \frac{1}{k} A_T
	\end{IEEEeqnarray*}
Here we have defined $A_T$ to be the sum of the individual accuracy rates.
We call  $(A \mid \bm{c})$ the posterior balanced accuracy rate, where
$\bm{c} =(c_1,...,c_k)$.
Most of the time, we simply want to calculate its expected value:
	\begin{IEEEeqnarray*}{lCl}
		\E{A \given \bm{c}} &=& \frac{1}{k} \, \E{A_T \given \bm{c}} \\
		&=& \frac{1}{k} \int a \cdot f_{A_T \mid \bm{c}}(a) \, da
	\end{IEEEeqnarray*}
Note that there is no closed form solution for the PDF $f_{A_T \mid \bm{c}}(a)$.
However assuming that $A_T$ is a sum of $k$ independent Beta random variables,
$f_{A_T \mid \bm{c}}(a)$ can be approximated by numerically convolving $k$ Beta distributions.
The independence assumption is reasonable here, since there should be little to no correlation
between the individual class accuracy rates. Knowing that a classifier is really good
at recognising stars does not tell us much about how well that classifier can recognise
galaxies.

Having the knowledge of $f_{A \mid \bm{c}}(a)$ will allow us to make violin plots,
construct confidence intervals and do hypothesis tests. To get an expression for this,
let us first rewrite the CDF as
	\begin{IEEEeqnarray*}{lCl}
		F_{A\mid \bm{c}}(a) &=& \Prob{A \leq a \mid \bm{c}} \\
		&=& \Prob[\Big]{\frac{1}{k} A_T \leq a \given \bm{c}} \\
		&=& \Prob{A_T \leq ka \given \bm{c}} \\
		&=& F_{A_T \mid \bm{c}}(ka) \IEEEyesnumber \label{eqn:CDF}
	\end{IEEEeqnarray*}
Differentiating \eqref{eqn:CDF} with respect to $a$, we obtain the PDF of $(A \mid \bm{c})$:
	\begin{IEEEeqnarray*}{lCl}
		f_{A \mid \bm{c}}(a) &=& \frac{\partial}{\partial a} F_{A \mid \bm{c}}(ka) \\
		&=& \frac{\partial}{\partial a} (ka) \cdot \frac{\partial}{\partial ka} F_{A_T \mid \bm{c}}(ka) \\
		&=& k \cdot f_{A_T \mid \bm{c}}(ka)
	\end{IEEEeqnarray*}
The posterior balanced accuracy rate is used throughout the experiments to report
the overall performance of a classifier. 

%%% Local Variables: 
%%% mode: latex
%%% TeX-master: "thesis"
%%% End: 

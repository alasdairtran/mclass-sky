
%%
%% Template chap2.tex
%%

\chapter{Photometric Classification}
\label{cha:ml}

Let us now bring machine learning to the realm of astronomy. We start with a motivation for why
astronomers might find machine learning helpful (Section \ref{sec:machine}) and an overview of
three families of classifiers (Section \ref{sec:classifiers}). We then discuss active learning and
six heuristics that can be used to rank unlabelled examples (Section \ref{sec:active} and
\ref{sec:heuristics}). Our novel contributions include the application of Thompson sampling to the
heuristic selection setting  (Section \ref{sec:bandit} and \ref{sec:thompson}) and a derivation of
the multi-class posterior balanced accuracy (Section \ref{sec:measures}) which can be used to
measure the performance of our algorithms.


\section{Maching Learning in Astronomy}
\label{sec:machine}

The two most common types of celestial objects are stars and galaxies. There are also some other
interesting objects such as quasars and white dwarfs. Quasars are thought to be supermassive black
holes surrounded by an accretion disc. They are very luminous and, unlike galaxies, appear as
single-source objects. White dwarfs are low to intermediate mass stars that are in their
final evolutionary stage. They are very dense and have a faint luminosity.

One way to classify objects into these various groups is to manually inspect their spectra. There
have even been attempts to make the process more automatic. For example, \citeN{hala14} achieved a
95\% accuracy rate by training a convolutional neural network on one-dimensional spectra to classify
objects into stars, quasars, and galaxies. Even so, it is currently not possible to obtain a
spectrum of every object, especially faint ones. This means only a small number of objects (e.g.
0.35\% in the SDSS dataset) can be classified this way. For the rest, we only have
photometric measurements.

Fortunately, the field of machine learning came about to solve this kind of problem. In the most
basic set-up, we have a collection of objects, each with a vector of photometric measurements $\bm{x} \in
\X$. A subset of them has been spectroscopically classified into some class $y \in \Y$ and they
form the labelled set $\Labelled \subset \X \times \Y$. We call $\X$ the feature space and $\Y$ the
label space. During the training phase, we feed a set of labelled examples, which we call the
training set, to a classifier. The classifier will then attempt to form a hypothesis $h: \X
\rightarrow \Y$. The labelling process might not be perfect, so the goal in machine learning is to
capture as much of the underlying trend in the data as possible, while avoiding fitting the random
noise. To see how well the hypothesis generalises to unknown data, we set aside a small number of
labelled examples which the classifier has not seen before. We call this the test set, and during
the testing phase, the true labels of test examples are compared to the predictions made by the
classifier.


% % % % % % % % % % % % % % % % % % % % % % % % % % % % % % % % % % % % % % % % % % % % % % % % % %
\section{Classifiers}
\label{sec:classifiers}

Three families of classifiers are used in our experiments. They are random forests, logistic
regression, and support vector machines. Below we give a quick overview of how each of them works.
We do not implement these classifiers ourselves. Rather we use scikit-learn \cite{pedregosa11}, the
most well-known machine learning package in Python.

\subsection{Random Forest}
\label{sub:forest}

To understand the motivation behind random forests, we first need to know how to construct
individual decision trees. Building each of these trees is like playing a game of Twenty Questions.
We start with the whole training set and at each step, we slice the feature space along the axis of
one feature. After many steps, we end up with a set of hyper-dimensional cuboids which form a
partition of the feature space. The algorithm stops when each of these cuboids contains data from
only one class. There are many criteria that we can use to decide on which feature and where we
slice along the axis. In this thesis, we use the Gini impurity which intuitively measures the
potential misclassification rate. In particular, let $k$ be the number of classes and $q_D(i)$ be
the frequency of objects belonging to class $i$ in set $D$. Then the Gini impurity of $D$ is the
probability that a randomly selected object from $D$ is misclassified if it were labelled according
to its frequency in $D$:
	\begin{IEEEeqnarray*}{lCl}
		\iota_G(D) &=& \sum_{i=1}^{k} q_D(i) (1 - q_D(i)) \\
		           &=& \sum_{i=1}^{k} q_D(i)  - \sum_{i=1}^{k} q_D(i)^2 \\
		           &=& 1 - \sum_{i=1}^{k} q_D(i)^2
	\end{IEEEeqnarray*}
When we partition $D$ into subsets $\{D_1, D_2, ..., D_d\}$, the Gini impurity of $D$ is now the sum
of the individual Gini impurities, weighted by the size of the subsets:
	\begin{IEEEeqnarray*}{lCl}
		\iota_G(D) &=& \sum_{i=1}^{d} \frac{|D_i|}{|D|} \iota_G(D_i)
	\end{IEEEeqnarray*}
Observe that if a subset contains objects from only one class, then its Gini impurity will be zero.
This gives us the following splitting criterion: at each step, slice the feature space along the
axis that will result in the greatest drop in the Gini impurity.

One problem with decision trees is that they tend to overfit the data and thus do not generalise
well. To solve this, \citeN{breiman01} proposes that we build many decision trees, thus creating a
random forest. The random forest makes its prediction by simply counting up the predictions of all
the individual trees and then taking the most popular choice. By averaging the predictions, we
avoid the problem of overfitting. Furthermore, for each tree, we only give it a small bootstrap
sample and at each split, we only consider a small number of features. This bootstrapping and
random subspace selection have been shown empirically to improve the accuracy rate \cite{breiman96,
ho98, louppe12}. Another nice feature of random forests is that they are extremely fast and hence
scale well with large datasets. Although they do not provide class probability estimates, we can
use proportions of the votes as a proxy for these probabilities.

\subsection{Logistic Regression}
\label{sub:logistic}

If we want the hypothesis to model actual class probabilities, then an alternative approach is to use
logistic regression. Developed by \citeN{cox58}, the algorithm tries to directly model the
probability of being in a class. Let $\bm{x}$ be the feature vector and $\bm{\beta}$ be the vector
of coefficients. The linear predictor $\eta$ is defined as
	\begin{IEEEeqnarray*}{lCl}
		\eta(\bm{x}) &=& \bm{\beta}^T \bm{x}
	\end{IEEEeqnarray*}	
Since probabilities must lie between 0 and 1, we want our predictor to have the same range. This can
be achieved by wrapping $\eta$ around the logistic function:
	\begin{IEEEeqnarray*}{lCl}
		p(y=1 | \bm{x}; \bm{\beta}) &=& \sigma(\eta(\bm{x}))  \\
                                    &=& \frac{1}{1 + e^{-\eta(\bm{x})}}
	\end{IEEEeqnarray*}
We can now interpret $p(y=1 | \bm{x}; \bm{\beta})$ as the probability that an object with feature vector
$\bm{x}$ belongs to the positive class. The goal of the algorithm is then to use the training data
to estimate the coefficient vector $\bm{\beta}$. This can be done by finding $\hat{\bm{\beta}}$ that
maximises the log-likelihood function:
    \begin{IEEEeqnarray*}{lCl}
        \hat{\beta} &=& \argmax_{\bm{\beta}} \sum_{i = 1}^{n} \log p(y_i| \bm{x}_i; \bm{\beta})
                         - \frac{1}{C} R(\bm{\beta})
    \end{IEEEeqnarray*}
where $n$ is the size of the training set, $R(\bm{\beta})$ is the regularisation term, and $C$ is
the inverse of the regularisation strength. A low value of $C$ forces the values of the parameters
to be small, thus avoiding overfitting. However if $C$ is too low, we might a hypothesis that is
too simple. For the regularisation term, we can either sum up the absolute values of the
coefficients (L1) or the squares of the coefficients (L2):
    \begin{IEEEeqnarray*}{lClllCl}
        R_{L1}(\bm{\beta}) &=& \sum_{i = 1}^{m} \abs{\bm{\beta}_i } &\qquad\qquad
        R_{L2}(\bm{\beta}) &=& \sum_{i = 1}^{m} \bm{\beta}_i^2
    \end{IEEEeqnarray*}
where $m$ is the number of features. One advantage of the L1 regularisation is that it leads to
sparse solutions, where a lot of coefficients become zero \cite{tibshirani96}. This is useful if we
have many features, which for example is the case after we do a polynomial transformation (see
Section \ref{sub:complex}).

There are a few ways to extend the above model to the multi-class setting. One option is
multinomial logistic regression, where we would need to jointly solve a set of $(k-1)$ binary
regressions if we have $k$ classes. In practice, when running the multinomial option in
scikit-learn, the probability estimates are not very reliable, especially when we have many
classes. The cause of this is unknown, but it is more likely due to flaws in the scikit-learn
implementation than in the actual theory.\footnote{See Appendix \ref{sec:forest_prob} for a plot
    comparing the learning curves of multinomial logistic regression with the one-vs-rest strategy.} A
more empirically stable alternative is to use the one-vs-rest strategy, where we run $k$
independent binary regressions. In particular, for class $i$, we pretend that the dataset contains
only objects from class $i$ and class `not $i$'. We then train the binary logistic model on this
simplified dataset and we do this $k$ times, one for each class. At the end, we end up with $k$
probabilities. These can be interpreted like normal class probabilities after normalisation.


\subsection{Support Vector Machines}
\label{sub:svm}

Support vector machines (SVM), first introduced by \citeN{boser92}, are another popular family of
algorithms. They have been used in astronomy, for example by \citeN{elting08}, to find non-linear
decision boundaries in the colour space of the SDSS dataset. The idea here is to find a decision
boundary that can maximise the distance between the boundary and the closest data points, which we
call support vectors. This involves finding the weights $\hat{\bm{w}}$ and the bias $\hat{b}$ that
minimise the objective function
    \begin{IEEEeqnarray*}{lCl}
        \argmin_{\bm{w}, b} R(\bm{w}) + C \sum_{i = 1}^{n} L(\bm{x}_i)
    \end{IEEEeqnarray*}
where $R(\bm{w})$ can either be L1 or L2 regularisation like in logistic regression, and
$L(\bm{x}_i)$ is the loss function. Two common loss functions are the hinge loss
    \begin{IEEEeqnarray*}{lCl}
        L(\bm{x}_i) &=& \max (0, 1 - y_i (\bm{w}^T \bm{x}_i + b))
    \end{IEEEeqnarray*}
and the square of the hinge loss
    \begin{IEEEeqnarray*}{lCl}
        L(\bm{x}_i) &=& \max (0, 1 - y_i (\bm{w}^T \bm{x}_i + b))^2
    \end{IEEEeqnarray*}
As usual, the penalty parameter $C$ controls the trade-off between the misclassification of
training examples and the simplicity of the decision surface, with a low value of $C$ resulting in
a simple hypothesis. SVMs, in their original formulation, are inherently binary classifiers.
However we can still use the one-vs-rest strategy to extend it to the multi-class setting. There
has even been an attempt to derive an inherently multi-class SVM \cite{crammer02}.


\subsection{Learning Complex Hypotheses}
\label{sub:complex}

Both SVMs and logistic regression are linear classifiers. When dealing with real-world data like
those in astronomy, we should not expect to be able to separate classes with a hyper-dimensional
plane. If we want them to learn more complex hypotheses, one option is do an explicit polynomial
transformation of the original photometric measurements. When we give the classifier the transformed
features, it would still find a linear boundary in the transformed space. However, the boundary
would mostly be non-linear in the original space.

A second option is to use the kernel trick which does an implicit map into a high-dimensional
feature space. For example, a popular kernel that is often used with SVMs is the radial basis
function (RBF), which actually maps the inputs into an infinite-dimensional space. The RBF kernel
has the parameter $\gamma$ which is inversely proportional to the radius of influence of the
support vectors. This means that a low value of $\gamma$ corresponds to a smoother model.

In random forests, we do not have to worry about any transformation. Although in each round, we
slice the data along only one axis, there is no limit on how many slices we can take and how small
the resulting cuboids can be. This allows us to learn arbitrarily complex hypotheses.


% % % % % % % % % % % % % % % % % % % % % % % % % % % % % % % % % % % % % % % % % % % % % % % % % %
\section{Overview of Active Learning}
\label{sec:active}

We now turn our attention to the construction of the training set. Getting spectroscopic labels is
expensive. Until now, astronomers do not have a quantitative method to help them choose objects
whose spectroscopic labels would provide the most amount of new information. Often, they simply
take a random sample of the sky. This, however, might not always be optimal. To see why, imagine
that there is a group of objects with very similar photometric measurements. We can obtain spectra
from all of them and conclude, for example, that they are all stars. However, a smarter way is to
get only one spectrum from this group for labelling and let the classifier generalise to other
similar objects. Keeping the size of the training set as small as possible while not sacrificing
the classifier accuracy is the goal of active learning.

There are three main types of active learning: membership query synthesis, stream-based selective
sampling, and pool-based active learning. In membership query synthesis, we are allowed to request
labels for any unlabelled instance in the feature space \cite{angluin88}. Equivalently, we may
request the astronomer to find an object with a certain combination of colours and magnitudes. This
is not very realistic since such objects might not even exist. In stream-based selective sampling,
we sample objects from the source one at a time, and as objects are streaming in, we must decide to
either label or discard each of them \cite{cohn94}. The assumption here is that it is free to obtain
unlabelled examples, which again is not applicable to astronomy. Thus we shall not discuss membership
query synthesis and stream-based selective sampling further in this thesis.

Instead, we shall focus on pool-based active learning \cite{lewis94}, the most relevant type of
active learning for astronomy. In this setting, we keep track of two sets. The labelled set
$\Labelled \subset \X \times \Y$ contains all examples that have been labelled by an expert so
far.\footnote{To keep it simple, we assume that all of $\Labelled$ is used as training data,
    and when we need to calculate the generalisation error, we shall withhold some labelled examples in
    $\Labelled$ from the classifier and use them as the test set.} All the remaining unlabelled
examples form the unlabelled set $\Unlabelled \subset \X$. The question now is how to select the
next training example from $\Unlabelled$. In practical terms, where should we next point the
telescope to, in order to obtain a spectrum? To answer this question, we need a rule $r(\bm{x}; h)$
that can assign a score to each object, based solely on their photometric features $\bm{x}$ and the
current hypothesis $h$. This score should reflect the amount of new information we would gain if we
were to label the object. Once we have computed $r(\bm{x}; h)$ for all candidates, we can then pick
the example with the highest score and obtain its spectrum. The object's feature vector and its
label are then added to the training set and the classifier is retrained to obtain a new $h$.

Finding an algorithm to compute $r(\bm{x}; h)$ exactly is still an open problem. or now, the best
that we can do is to come up with heuristics that can approximate $r(\bm{x}; h)$. Another problem
is that in practice, the unlabelled pool can be arbitrarily large. For example, there are 800
million unlabelled objects in the SDSS. Thus if we only have a limited computing power, in each
round, we might only be able to assign scores to a subset $E \subseteq \Unlabelled$ of size $t$. A
formal description of the active learning routine is given as follows. As we shall see in Section
\ref{sec:heuristics}, for some active learning heuristics, we need to substitute $\argmax$ with
$\argmin$ in line 4 of Algorithm \ref{alg:active}.

\algblock[Name]{Start}{End}

\algblockdefx[Forall]{Foreach}{Endforeach}%
			[1]{\textbf{for each} #1 \textbf{do}}%
			{\textbf{end for}}

\begin{algorithm}[h]
	\caption{The general pool-based active learning algorithm}
	\label{alg:active}
	\begin{algorithmic}[1]
		\Procedure {ActiveLearner}{$\Unlabelled$, $\Labelled$, $h$, $r$, $n$, $t$}
			\While {$|\Labelled| < n$}
				\State $E$ $\leftarrow$ random sample of size $t$ from $\Unlabelled$
				\State $\bm{x}_* \leftarrow \argmax_{\bm{x} \in E} r(\bm{x}; h)$
				\State $y_* \leftarrow$ ask the expert to label $\bm{x}_*$
				\State $\Labelled \leftarrow \Labelled  \cup (\bm{x}_*, y_*)$
				\State $\Unlabelled \leftarrow \Unlabelled \setminus \bm{x}_*$
				\State $h_\Labelled(\bm{x}) \leftarrow$ retrain the classifier
			\EndWhile
			\EndProcedure
	\end{algorithmic}
\end{algorithm}


% % % % % % % % % % % % % % % % % % % % % % % % % % % % % % % % % % % % % % % % % % % % % % % % % %
\section{Active Learning Heuristics}
\label{sec:heuristics}

Many methods have been proposed to rank the informativeness of unlabelled objects. The four
prominent families of heuristics are uncertainty sampling, version space reduction, loss function
minimisation, and classifier certainty. All of these heuristics require the class probabilities
estimated by the current hypothesis. We now discuss each of them in turn, starting with the least
computationally expensive one.

\subsection{Uncertainty Sampling}
\label{sub:uncertainty}

\citeN{lewis94} introduce the idea of uncertainty sampling, where we select the example whose class
membership the classifier is least certain about. These tend to be points that are near the
decision boundary of the classifier. One way to quantify the uncertainty is to calculate the
entropy \cite{shannon48}, which measures the amount of information needed to encode a distribution.
Intuitively, the closer class probabilities of an object are to random guessing, the higher its
entropy will be. This gives us the heuristic of picking the candidate with the highest entropy:
	\begin{IEEEeqnarray*}{lCl}
        \bm{x}_*
        &=&  \argmax_{x \in E} r_S(\bm{x}; h) \\
        &=&  \argmax_{x \in E} \left\{-\sum_{y \in \Y} p(y | \bm{x}; h)
        \log \big[ p(y | \bm{x}; h) \big] \right\}
    \end{IEEEeqnarray*}
In fact, if we care about how close the class probabilities are to random guessing, there is an
even simpler measure. \shortciteN{scheffer01} define the margin as the difference between the two
highest class probabilities. Since the sum of all probabilities must be 1, the smaller the margin
is, the more uncertain we are about its class membership. Thus another heuristic to pick the
candidate with the smallest margin:
	\begin{IEEEeqnarray*}{lCl}
        \bm{x}_*
        &=& \argmin_{x \in E} r_M(\bm{x}; h)  \\
		&=& \argmin_{x \in E} \left\{ \max_{y \in \Y} p(y | \bm{x}; h) -
            \max_{z \in \Y \setminus \{y\}} p(z | \bm{x}; h)  \right\}
	\end{IEEEeqnarray*}


%--------------------------------------------------------------------------------------------------
\subsection{Version Space Reduction}
\label{sub:qbb}

Let us define the version space as the set of all possible hypotheses that are consistent with the
current training set. Instead of focussing on the uncertainty of individual predictions, we could
instead try to constrain the size of the version space, allowing the search for the optimal
hypothesis to be more precise. To quantify the size the version space, we can train a committee of
classifiers, $\B = \{h_1, h_2, ..., h_B\}$, and measure the disagreement among the members about an
object's class membership. Each committee member needs to have a a hypothesis that is as different
from the others as possible but that is still in the version space \cite{melville04}. In order to
have this diversity, we give each member only a subset of the training examples. Since there might
not be enough training data (for example, in our experiments, we have 11 members but only a maximum
of 300 labelled points), we need to use bootstrapping and select samples with replacement. Hence
this method is often called Query by Bagging (QBB).

One way to measure the level of disagreement is to calculate the margin using the class
probabilities estimated by the committee \citeN{melville04}. This looks similar to one of the
uncertainty sampling heuristic, except now we first average out the probabilities of the members
before minimising the margin:
    \begin{IEEEeqnarray*}{lCl}
        \bm{x}_*
        &=& \argmin_{x \in E} r_{QM}(\bm{x}; \B)  \\
        &=& \argmin_{x \in E} \left\{ \max_{y \in \Y} p(y | \bm{x}; \B) -
        \max_{z \in \Y \setminus \{y\}} p(z | \bm{x}; \B)  \right\}
    \end{IEEEeqnarray*}
where
	\begin{IEEEeqnarray*}{lCl}
		p(y | \bm{x}; \B) &=& \dfrac{1}{B} \sum_{b=1}^{B} p(y | \bm{x}, h_b)
	\end{IEEEeqnarray*}
In addition to the margin, \citeN{mccallum98} offer an alternative disagreement measure which
involves picking the candidate with the largest expected Kullback--Leibler (KL) divergence
from the average:
	\begin{IEEEeqnarray*}{lCl}
        \bm{x}_*
        &=& \argmax_{x \in E} r_{QK}(\bm{x}; \B)  \\
		&=& \argmax_{x \in E} \left\{ \dfrac{1}{B} \sum_{b=1}^B D_{\mathrm{KL}}(p_b\|p_\B) \right\}
	\end{IEEEeqnarray*}
where 
	\begin{IEEEeqnarray*}{lCl}
		D_{\mathrm{KL}}(p_b\|p_\B) = \sum_{y \in \Y} p(y | \bm{x}; h_b) \,
		                             \ln\frac{p(y | \bm{x}; h_b)}{p(y | \bm{x}; \B)}
	\end{IEEEeqnarray*}
The KL divergence measures the amount of information lost when $p_\B$ is used to approximate $p_b$.
Intuitively, the larger the KL divergence is, the more disagreement there is between $p_\B$ and
$p_b$. In the active learning context, $p_\B$ is the average prediction probability distribution of
the committee, while $p_b$ is the prediction of a particular committee member.


%--------------------------------------------------------------------------------------------------
\subsection{Loss Function Minimisation}
\label{sub:variance}

The third approach involves minimising a loss function directly, which in turn will minimise the
future generalisation error. A commonly used loss function is the squared loss that has the
following decomposition:
	\begin{IEEEeqnarray*}{lCl}
		\E{\text{Squared Loss}} &=& \text{Bias}^2 + \text{Variance} + \text{Noise}
	\end{IEEEeqnarray*}
Since the noise is intrinsic to the data and represents the expected loss under the optimal
hypothesis, there is nothing we can do about it. The squared bias reflects the error due to the model
class itself. For example, there will be bias if we use a linear hypothesis to learn a non-linear
function. Thus the bias is fixed under the same classifier. However, under certain assumptions
like the consistency of parameter estimates, the variance will vanish as the training set size
approaches infinity. This gives us the heuristic of picking the candidate that would cause the
greatest drop in the variance if we knew its label. Unfortunately, this is a chicken-and-egg problem
since we need to know the labelling information before we can calculate the drop in the variance,
which defeats the purpose of the approach. The next best thing we can do is to pick the candidate
that will result in the lowest expected variance:
    \begin{IEEEeqnarray*}{lCl}
        \bm{x}_*
        &=& \argmin_{x \in E} r_V(\bm{x}; h)  \\
        &=& \argmin_{x \in E}  \E{\text{V}(\Labelled \cup (\bm{x}, y))} \\
        &=& \argmin_{x \in E} \left\{ \sum_{y \in \Y} p(y | \bm{x}; h)
            \text{V}(\Labelled \cup (\bm{x}, y); h)  \right\}
    \end{IEEEeqnarray*}
where the expectation is over the class probability distribution under the current hypothesis and
$\text{V}(\Labelled \cup (\bm{x}, y); h)$ is the variance of the model after $(\bm{x}, y)$ has been
added to the label set $\Labelled$. Note that this is quite an expensive computation, since to
assign a score to each candidate, we first need to give it each of the possible labels, add it to
the training set to get an updated hypothesis, and calculate the new variance.

In addition, estimating $\text{V}(\Labelled; h)$ requires a bit of work. In
multinomial logistic regression, the hypothesis maps features directly to class probabilities.
This allows \citeN{schein07} to take the first two terms of the Taylor series expansion
of the probability
	\begin{IEEEeqnarray*}{lCl}
		p(y | \bm{x}, \bm{\hat{w}}, h)
		&\approx& p(y | \bm{x}, \bm{w}, h) + \mathbf{g}_{\bm{x}}(y)(\bm{\hat{w}} - \bm{w})
	\end{IEEEeqnarray*}
where $\bm{w}$ and $\bm{\hat{w}}$ are the expected and the current estimates of the model
parameters, respectively, and $\mathbf{g}_{\bm{x}}(y)$ is called the gradient vector. Let $F$
be the Fisher information matrix and
	\begin{IEEEeqnarray*}{lCl}
		A &=& \sum_{\bm{x} \in \Unlabelled}
		      \sum_{y \in \Y} \mathbf{g}_{\bm{x}}(y) ~ \mathbf{g}_{\bm{x}}(y)^T
	\end{IEEEeqnarray*}
\citeN{schein07} show that
	\begin{IEEEeqnarray*}{lCl}
		\text{V}(\Labelled; h) &=& tr(AF^{-1})
	\end{IEEEeqnarray*}
where the trace function $tr(X)$ is the sum of the elements along the main diagonal of a square
matrix $X$. Note that the above expression is specific to multinomial logistic regression. If we
use the one-vs-rest strategy with binary logistic regression or another entirely different
classifier like SVMs in our experiments, the same approximation might not hold and we should not
expect to get good results. We leave the variance estimation of other learning algorithms for
future work.


%--------------------------------------------------------------------------------------------------
\subsection{Classifier Certainty}
\label{sub:cc}

Finally, instead of minimising the variance of the unlabelled pool, \citeN{mackay91} proposes
minimising the entropy of the classifier's predictions on $\Unlabelled$:
    \begin{IEEEeqnarray*}{lCl}
		CC(\Labelled; h) &=& - \sum_{\bm{x} \in \Unlabelled} \sum_{y \in \Y}
							 p(y | \bm{x}, h) \log \big[ p(y | \bm{x}, h)  \big]
	\end{IEEEeqnarray*}
Although this sounds
similar to one of the uncertainty sampling heuristics, here instead of picking the 
most uncertain candidate, we pick the candidate that is expected to increase the classifier's
prediction certainty by the the greatest amount:
	\begin{IEEEeqnarray*}{lCl}
        \bm{x}_*
        &=& \argmin_{x \in E} r_{CC}(\bm{x}; h) \\
        &=& \argmin_{x \in E}  \E{CC(\Labelled \cup (\bm{x}, y))} \\
		&=& \argmin_{x \in E} \left\{ \sum_{y\in \Y} p(y|\bm{x}; h) 
             CC(\Labelled \cup (\bm{x}, y); h) \right\}  
	\end{IEEEeqnarray*}
Like the variance minimisation technique, we need to find the expectation over the possible classes.
Thus to get the score of just one candidate, we need to retrain the classifier $k$ times, where
$k$ is the number of labels. In practice, both the variance and the classifier certainty
heuristics are too too computationally expensive to run.

\subsection{Summary of Heuristics}

\begin{table}[h]
	\caption {Summary of active learning heuristics used in our experiments} \label{tab:heuristics}
	\centering
	\begin{tabular}{lll}
		\toprule
		{Name}  & Notation &  Objective  \\
		\midrule
		Entropy & $r_S(\bm{x}; h)$
			& $\argmax_{x \in E} \left\{-\sum_{y \in \Y} p(y | \bm{x}; h)
            \log \big[ p(y | \bm{x}; h) \big] \right\}$
			\\[2ex]
		Margin & $r_M(\bm{x}; h)$
			& $\argmin_{x \in E} \left\{ \max_{y \in \Y} p(y | \bm{x}; h) -
            \max_{z \in \Y \setminus \{y\}} p(z | \bm{x}; h)  \right\}$
			\\[2ex]
		QBB Margin & $r_{QM}(\bm{x}; h)$
			& $\argmin_{x \in E} \left\{ \max_{y \in \Y} p(y | \bm{x}; \B) -
            \max_{z \in \Y \setminus \{y\}} p(z | \bm{x}; \B)  \right\}$
			\\[2ex]
		QBB KL & $r_{QK}(\bm{x}; h)$
			& $\argmax_{x \in E} \left\{ \dfrac{1}{B}
               \sum_{b=1}^B D_{\mathrm{KL}}(p_b\|p_\B) \right\}$
			\\[2ex]
		Pool Variance & $r_V(\bm{x}; h)$
			& $\argmin_{x \in E} \left\{ \sum_{y \in \Y} p(y | \bm{x}; h)
            \text{V}(\Labelled \cup (\bm{x}, y); h)  \right\}$
			\\[2ex]
		Pool Entropy & $r_{CC}(\bm{x}; h)$
			& $\argmin_{x \in E} \left\{ \sum_{y\in \Y} p(y|\bm{x}; h) 
               CC(\Labelled \cup (\bm{x}, y); h) \right\}  $
			\\
		\bottomrule
	\end{tabular}
\end{table}


% % % % % % % % % % % % % % % % % % % % % % % % % % % % % % % % % % % % % % % % % % % % % % % % % %
\section{Multi-arm Bandit}
\label{sec:bandit}

Out of the six heuristics discussed, how do we know which one is the optimal, anyway? There have
been some attempts in the literature to do a theoretical analysis of them. However proofs are
scarce, and when there is one available, they normally only work under simplifying assumptions. For
example, \shortciteN{freund97} show that the query by committee algorithm (a slight variant of our
QBB heuristics) guarantees an exponential decrease in the prediction error with the training size,
but only under certain restrictions such as there is no noise. Thus whether many of these are
guaranteed to beat random sampling is still an open question. We do not worry too much about
theoretical analysis in this thesis. Instead we focus on an empirical analysis in the astronomical
domain.

To help us automatically choose the optimal heuristic, 
we now turn our attention to the multi-armed bandit problem in probability theory. The colourful
name originates from the situation where a gambler stands in front a slot machine with $n$ levers.
When pulled, each lever gives out a random reward according to some unknown distribution.
The goal of the game is to come up with a strategy that can maximise the gambler's
lifetime rewards with a minimum number of pulls.

The key novel contribution of this thesis is the application of this theory to the problem of
heuristic selection. Suppose we have a set of $n$ heuristics $ \R = \{r_1, ..., r_n \}$. Each heuristic
has a different ability to pick the candidate that can give the biggest gain in information
when added to the training set. An appropriate reward is then the incremental increase
in the accuracy rate. Observe that the heuristic rewards are independent of each other,
since the theories with which we use to derive the heuristics are mostly unrelated.

Let $\bm{w}$ be the reward vector where entry $\bm{w}_i$ is the reward of heuristic $r_i$.
Observe that even with the optimal heuristic, there could be error during the labelling
process that causes the accuracy rate to decrease. Conversely, a bad heuristic might be
able to pick an informative candidate due to pure luck. Thus there is always a certain level
of randomness in the reward received. These errors are probably normally distributed, so
	\begin{IEEEeqnarray*}{lCl}
		(\bm{w} \mid \bm{\nu}, \tau^2) \sim N(\bm{\nu}, \tau^2 \bm{I})
	\end{IEEEeqnarray*}
where $\bm{I}$ is the identity matrix.
To make the problem tractable, assume that we know the constant variance $\tau^2$. Assume that
the mean vector $\bm{\nu}$ follows a normal distribution
	\begin{IEEEeqnarray*}{lCl}
		\bm{\nu} \sim N(\bm{\mu}, \sigma^2 \bm{I})
	\end{IEEEeqnarray*}
Since we do not yet have any information about the performance of each heuristic,
the hyperparameters $\bm{\mu}$ and $\sigma^2$ are unknown.

One problem in multi-arm bandits is the trade-off between exploration and exploitation. Suppose we
have managed to estimate $\bm{\mu}$ and $\sigma^2$, then by always selecting the heuristic with the
highest possible $\bm{\mu}$, or the greedy heuristic, we would be exploiting our current knowledge.
If however we select one of the other non-greedy heuristics, we would then be exploring with the
intention of improving our current estimates of $\bm{\mu}$ and $\sigma^2$. There are many instances
in which we find our previously held beliefs to be completely wrong. Thus by always exploiting, we
might miss out on the optimal heuristic. On the other hand, if we explore too much, it might take a
long time to reach the desired accuracy rate and the strategy ends up being no different from
random sampling. \info{relevant to ASKAP and to us anyway}

\section{Thompson Sampling}
\label{sec:thompson}

There are two main methods in the literature that address this exploration vs exploitation 
problem. The algorithm with a strong theoretical guarantee is Upper Confidence Bound
\cite{auer02}. We shall, however, focus on a simpler and much older algorithm called Thompson sampling.
First introduced by \citeN{thompson33}, the algorithm solves the trade-off from
a Bayesian perspective. It has been shown to achieve results
that are comparable and sometimes even better than Upper Confidence Bound \cite{chapelle11}.

Under the heuristic selection setting, Thompson sampling works as follows.
We start with a prior knowledge of $\bm{\mu}$ and $\sigma^2$. In each
round, we draw a random sample from the distribution $N(\bm{\mu}, \sigma^2 \bm{I})$
and select heuristic $r_*$ that has the highest mean reward value in the sample. We then 
observe the actual reward $w_{*}$ received and use it to update the prior accordingly.
Under the normal likelihood, the normal distribution is conjugate to itself, and it can be
shown that the posterior distribution of the mean reward $\bm{\nu}_{*}$
remains normal, but now with parameters
	\begin{IEEEeqnarray*}{lCl}
		(\bm{\nu}_{*} \mid w_{*}) \sim N \left(
		\frac{\bm{\mu}_* \bm{\tau}_* + w_{r^*} \bm{\sigma}_*}{\bm{\tau}_* + \bm{\sigma}_*},
		\frac{\bm{\sigma}_* \bm{\tau}_*}{\bm{\tau}_* + \bm{\sigma}_*}
		\right)
	\end{IEEEeqnarray*}
Below is the formal specification of the algorithm.

\begin{algorithm}[h]
	\caption{Thompson sapmling}
	\label{alg:thompson}
	\begin{algorithmic}[1]
		\Procedure {ThompsonSampling}{$\R$, $\bm{\mu}$, $\bm{\sigma}$, $\bm{\tau}$, n}
		\Foreach {$t \in \{1, 2, ..., n\}$}
		\Foreach {$r \in \R$}
		\State $\bm{\nu}_r' \leftarrow$ draw a sample from $N(\bm{\mu_r}, \bm{\sigma_r})$
		\Endforeach
		\State $r_* \leftarrow \argmax_{r \in \R}\bm{\nu}_r'$
		\State Observe reward $w_{*}$
		\State Update $\bm{\mu}_{*}$
		\State Update $\bm{\sigma}_{*}$
		\Endforeach
		\EndProcedure
	\end{algorithmic}
\end{algorithm}

If we combine Algorithm \ref{alg:active} and \ref{alg:thompson}, we end up with
an algorithm that can automatically select the optimal active learning heuristic
with Thompson sampling.

\begin{algorithm}[h]
	\caption{The multi-arm bandit active learning algorithm}
	\label{alg:bandit}
	\begin{algorithmic}[1]
		\Procedure {ActiveBandit}{$\Unlabelled$, $\Labelled$, $h$, $n$, $t$,
			                      $\R$, $\bm{\mu}$, $\bm{\sigma}$, $\bm{\tau}$}
		\While {$|\Labelled| < n$}
		\Foreach {$r \in \R$}
			\State $\bm{\nu}_r' \leftarrow$ draw a sample from $N(\bm{\mu_r}, \bm{\sigma_r})$
		\Endforeach
		\State $r_* \leftarrow \argmax_{r \in \R} \bm{\nu}_r'$
		\State $E$ $\leftarrow$ random sample of size $t$ from $\Unlabelled$
		\State $\bm{x}_* \leftarrow \argmax_{\bm{x} \in E} r_*(\bm{x})$
		\State $y_* \leftarrow$ ask the expert to label $\bm{x}_*$
		\State $\Labelled \leftarrow \Labelled  \cup (\bm{x}_*, y_*)$
		\State $\Unlabelled \leftarrow \Unlabelled \setminus \bm{x}_*$
		\State $h_\Labelled(\bm{x}) \leftarrow$ retrain the classifier
		\State $\delta$ $\leftarrow$ change in the balanced accuracy rate 
		\State Update $\bm{\mu}_{*}$
		\State Update $\bm{\sigma}_{*}$
		\EndWhile
		\EndProcedure
	\end{algorithmic}
\end{algorithm}

As we shall see in Chapter \ref{cha:expt2}, the reward function dynamically evolves
as the training size increases. Intuitively, the accuracy rate can never go beyond 100\%, so
we would expect the incremental change in the accuracy rate to become smaller over time.
Attempts to address this problem have been made in the literature. For example \citeN{gupta11}
introduce the Dynamic Thompson Sampling method that manages to adapt to the evolving
parameters faster than Algorithm \ref{alg:thompson}. However, we shall leave the investigation
of such methods to future work.


% % % % % % % % % % % % % % % % % % % % % % % % % % % % % % % % % % % % % % % % % % % % % % % % % % 
\section{Performance Measures}
\label{sec:measures}

We end this chapter with a description of two performance measures that are used in the experiments.

\begin{figure}[tbp]
	\centering
	\renewcommand\arraystretch{1.5}
	\setlength\tabcolsep{0pt}
	\begin{tabular}{c >{\bfseries}r @{\hspace{0.7em}}c @{\hspace{0.4em}}c @{\hspace{0.4em}}c}
		\multirow{13}{*}{\rotatebox{90}{\parbox{1.1cm}{\bfseries\raggedleft Actual}}} & 
		& \multicolumn{3}{c}{\bfseries Predicted} \\
		& & \bfseries Galaxy & \bfseries Star & \bfseries Quasar \\
		& Galaxy & \MyBox{97,621}{95.7\%} & \MyBox{492}{0.5\%} & \MyBox{1,887}{1.8\%} \\[2.4em]
		& Star & \MyBox{1,625}{1.6\%} & \MyBox{89,489}{95.4\%}  & \MyBox{8,886}{8.5\%} \\[2.4em]
		& Quasar & \MyBox{2,790}{2.7\%} & \MyBox{3,868}{4.1\%}  & \MyBox{93,342}{89.7\%}
	\end{tabular}
	\caption[Confusion matrix of random forest on SDSS]{
		The confusion and the normalised confusion matrix of the random forest
		on the SDSS test set. For example, out of all the objects predicted as quasars, 8.5\%
		of them are actually stars. Out of all the objects predicted as stasr, 4.1\% of them
		are actually quasars.}
	\label{fig:recall}
\end{figure}

\subsection{Recall}
\label{sub:recall}

Let $C$ be the confusion matrix of a classifier on a test set,
where entry $C_{ij}$ is the number of objects in class $i$
but have been predicted to be in class $j$.
Recall measures the classifier's ability to find all the positive examples and avoid
having false negatives:
\begin{IEEEeqnarray*}{lCl}
	\text{Recall}_i &=& \frac{C_{ii}}{\sum_j C_{ij}}
\end{IEEEeqnarray*}
We use recall to plot the the performance of a classifier on the celestial sphere. This
measure is chosen mainly due to its simple implementation.

\subsection{Posterior Balanced Accuracy Rate}
\label{sub:pba}

Certain astronomical objects are either rarer or more difficult to detect than others. In the SDSS
labelled set, there are 4.5 times as many galaxies as quasars. The problem of class imbalance is
even more severe in the VST-ATLAS set, with 11 times more stars than white dwarfs. An easy fix is
to undersample the dominant class when creating training and test sets. This, of course, means that
the size of these sets are limited by the size of the minority class.

When we do not want to alter the underlying class distributions or when larger training and test
sets are desired, we need a performance measure that can correct for the class imbalance.
\shortciteN{brodersen10} showed that the posterior balanced accuracy distribution can overcome the
bias in the binary case. We now extend this idea to the multi-class setting.

Suppose we have $k$ classes. For each class $i$ between $1$ and $k$, there are $N_i$ objects in the
universe. Given a classifier, we can assign a predicted label to every object and compare our
prediction to the true label. Let $C_i$ be the number of objects in class $i$ that are correctly
predicted. Then we define the accuracy rate $A_i$ of class $i$ as
	\begin{IEEEeqnarray*}{lCl}
		A_i &=& \frac{C_i}{N_i}
	\end{IEEEeqnarray*}
Initially we have no information about $C_i$ and $N_i$, so we can assume that each $A_i$ 
follows a uniform prior from 0 to 1. This is the same as a Beta distribution
with parameters $\alpha = \beta = 1$:
	\begin{IEEEeqnarray}{lCl}
		A_i &\sim& \Beta(1,1) \label{eqn:prior}
	\end{IEEEeqnarray}
After we have trained the classifier, suppose we have a test set containing $n_i$
objects in class $i$. Running the classifier on this test set is the same as conducting
$k$ binomial experiments, where, in the $i$th experiment, the sample size is
$n_i$ and the probability of success is simply the accuracy rate $A_i$. Let $c_i$ be
the number of correctly labelled objects belonging to class $i$ in the test set. Then,
conditional on the accuracy rate, $c_i$ follows a binomial distribution:
	\begin{IEEEeqnarray}{lCl}
		(c_i \mid A_i) &\sim& \Bin(n_i, A_i) \label{eqn:likelihood}
	\end{IEEEeqnarray}
In the Bayesian setting, \eqref{eqn:prior} is the prior and \eqref{eqn:likelihood}
is the likelihood. To get the posterior PDF, we simply multiply the prior with the likelihood:
	\begin{IEEEeqnarray*}{lCl}
		f_{A_i \mid \bm{c}}(a)
		&\propto& f_{A_i}(a) \times f_{c_i \mid A_i}(c_i) \\
		&\propto& a^{1-1}(1-a)^{1-1} \times a^{c_i} (1 - a)^{n_i - c_i} \\
		&=& a^{1 + c_i - 1}(1-a)^{1 + n_i - c_i - 1}
	\end{IEEEeqnarray*}
Thus, with respect to the binomial likelihood function,
the Beta distribution is conjugate to itself. The posterior accuracy rate $A_i$
also follows a Beta distribution, now with parameters
	\begin{IEEEeqnarray*}{lCl}
		(A_i \mid c_i) &\sim& \Beta(1 + c_i, 1 + n_i - c_i)
	\end{IEEEeqnarray*}
Recall that our goal is to have a balanced accuracy rate, $A$, that puts an equal
weight in each class. This can be achieved by taking the average of all the class accuracy rates:
	\begin{IEEEeqnarray*}{lCl}
		A &=& \frac{1}{k} \sum_{i=1}^k A_i \\
		&=& \frac{1}{k} A_T
	\end{IEEEeqnarray*}
Here we have defined $A_T$ to be the sum of the individual accuracy rates.
We call  $(A \mid \bm{c})$ the posterior balanced accuracy rate, where
$\bm{c} =(c_1,...,c_k)$.
Most of the time, we simply want to calculate its expected value:
	\begin{IEEEeqnarray*}{lCl}
		\E{A \given \bm{c}} &=& \frac{1}{k} \, \E{A_T \given \bm{c}} \\
		&=& \frac{1}{k} \int a \cdot f_{A_T \mid \bm{c}}(a) \, da
	\end{IEEEeqnarray*}
Note that there is no closed form solution for the PDF $f_{A_T \mid \bm{c}}(a)$.
However assuming that $A_T$ is a sum of $k$ independent Beta random variables,
$f_{A_T \mid \bm{c}}(a)$ can be approximated by numerically convolving $k$ Beta distributions.
The independence assumption is reasonable here, since there should be little to no correlation
between the individual class accuracy rates. Knowing that a classifier is really good
at recognising stars does not tell us much about how well that classifier can recognise
galaxies.

Having the knowledge of $f_{A \mid \bm{c}}(a)$ will allow us to make violin plots,
construct confidence intervals and do hypothesis tests. To get an expression for this,
let us first rewrite the CDF as
	\begin{IEEEeqnarray*}{lCl}
		F_{A\mid \bm{c}}(a) &=& \Prob{A \leq a \mid \bm{c}} \\
		&=& \Prob[\Big]{\frac{1}{k} A_T \leq a \given \bm{c}} \\
		&=& \Prob{A_T \leq ka \given \bm{c}} \\
		&=& F_{A_T \mid \bm{c}}(ka) \IEEEyesnumber \label{eqn:CDF}
	\end{IEEEeqnarray*}
Differentiating \eqref{eqn:CDF} with respect to $a$, we obtain the PDF of $(A \mid \bm{c})$:
	\begin{IEEEeqnarray*}{lCl}
		f_{A \mid \bm{c}}(a) &=& \frac{\partial}{\partial a} F_{A \mid \bm{c}}(ka) \\
		&=& \frac{\partial}{\partial a} (ka) \cdot \frac{\partial}{\partial ka} F_{A_T \mid \bm{c}}(ka) \\
		&=& k \cdot f_{A_T \mid \bm{c}}(ka)
	\end{IEEEeqnarray*}
The posterior balanced accuracy rate is used throughout the experiments to report
the overall performance of a classifier. 

%%% Local Variables: 
%%% mode: latex
%%% TeX-master: "thesis"
%%% End: 

%%
%% Template intro.tex
%%

\chapter{Introduction}
\label{cha:intro}

The night sky has always been a source of wonder throughout history. Even before the Age of
Science, we used our built-in pattern recognition and tried to assign a purpose to the cosmos with
stories of gods and the supernatural. Then came the Ancient Greeks who brought mathematics to
astronomy and sought a more rational explanation of celestial bodies. But we could only discover so
much with our naked eyes. It was not until the invention of new technology during the Scientific
Revolution like the telescope that we began to make significantly more progress. And yet, 400 years
after Galileo made the first working telescope, most of the sky still remains unknown to us.

This, however, is starting to change. The Sloan Digital Sky Survey (SDSS) began in 2000 and has
since obtained 800 million sets of photometric measurements and 3 million spectra of the northern
sky. Similar projects are going on to map the southern sky including the VST ATLAS and the
SkyMapper surveys. We now face another challenge of how to analyse this huge amount of data.
Fortunately, we can bring machine learning to astronomy and let the computer to do the pattern
recognition task. This thesis is an attempt to study a small part of the interaction between these
two disciplines.

% % % % % % % % % % % % % % % % % % % % % % % % % % % % % % % % % % % % % % % % % % % % % % % % % %
\section{Contributions}
\label{sec:contributions}

The thesis centres around the problem of how we can use machine learning to choose objects whose
spectroscopic labels would be most informative to the task of photometric classification. Our key
novel contribution is the application of Thompson sampling, a Bayesian solution to the
exploration-exploitation trade-off, to the selection of six active learning heuristics. The
algorithm is described in Section \ref{sec:bandit} and \ref{sec:thompson}, and the experimental
results are discussed in Chapter \ref{cha:expt2}. As far as we know, this is the first time that
such active learning heuristics are applied to the astronomical domain. It is also the first time
that the multi-arm bandit problem with Thompson sampling is used in the heuristic selection
context. Along the way, we make four other contributions that would support the active learning
experiment. These are
	\begin{itemize}
		\item Comparing three sets of dust extinction vectors and seeing which one is the most
		effective at improving the accuracy (Section \ref{sub:extinction}).
		
		\item Identifying good photometric features using both well-known machine learning techniques
		like polynomial transformation and domain knowledge such as the use of colours (Section
		\ref{sec:prep} and \ref{sub:hyper}).
		
		\item Training three families of classifiers to do photometric classification and using
		a random forest to predict the class proportion of unlabelled data (Section \ref{sub:lc}
		and \ref{sub:prop}).
		
		\item Deriving an extension of the posterior balanced accuracy to the multi-class
		setting, which will be used to evaluate the performance of our algorithms (Section 
		\ref{sec:measures}).
	\end{itemize}
An important part of this project involves creating an open-source, extendable, and well-documented
Python package that allows astronomers to perform active learning routines and make quick
visualisations of photometric data. The package and reproducible code of all experiments are
available on the project's GitHub repository\footnote{
	\url{https://github.com/alasdairtran/mclearn}}.


% % % % % % % % % % % % % % % % % % % % % % % % % % % % % % % % % % % % % % % % % % % % % % % % % %
\section{Thesis Outline}
\label{sec:orgnisation}

The thesis consists of four main chapters, two of which are dedicated to a discussion of ideas and
the other two focus on experiments:
	\begin{itemize}
		\item Chapter \ref{cha:astro} introduces the reader to the tools that astronomers use
		to map the sky. This includes important concepts like photometry, magnitudes, dust
		extinction, and the celestial coordinate system. We also give a brief overview of
		two datasets, SDSS and VST ATLAS, that are used in our experiments.
		
		\item Chapter \ref{cha:ml} introduces machine learning and surveys the relevant literature
		in pool-based active learning. This leads to a discussion of the multi-arm bandit problem
		and the use of Thompson sampling in the setting where each bandit arm is an active learning
		heuristic. We end the chapter with a derivation of multi-class posterior balanced accuracy,
		an important performance measure for data with unbalanced classes.
		
		\item Chapter \ref{cha:expt1} and \ref{cha:expt2} test the ideas discussed so far in a
		series of experiments on the SDSS and the VST-ATLAS datasets. A detailed protocol and a
		thorough discussion of the results are given for each experiment. In particular, we offer
		insights into the areas where Thompson sampling outperforms random sampling and where it
		underperforms. 
	\end{itemize}
For those interested in reproducing the experiments and applying the ideas in this thesis to their
own datasets, Appendix \ref{cha:datasets} provides information on how to obtain the SDSS dataset and
Appendix \ref{cha:mclearn} contains a guide on how to install and use the
accompanying Python package \texttt{mclearn}. Supplementary results are included in Appendix
\ref{cha:dustvectors} and \ref{cha:supp}.

%%% Local Variables: 
%%% mode: latex
%%% TeX-master: "thesis"
%%% End: 

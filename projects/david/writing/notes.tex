% Preamble
\documentclass[11pt]{amsart}
\usepackage{mathtools}
\usepackage{amssymb,latexsym}
\usepackage{physics}
\usepackage{listings}
\usepackage{bm}
\usepackage[margin=1.4in]{geometry}
\usepackage{enumerate}
\usepackage{graphicx}
\usepackage{color}
\usepackage{hyperref}
\hypersetup{
    colorlinks=true, % set true if you want colored links
    linktoc=all,     % set to all if you want both sections and subsections 
                     % linked
    linkcolor=blue,  % choose some color if you want links to stand out
}
\usepackage{parskip}

% Environments
\theoremstyle{definition}
\newtheorem{algorithm}{Algorithm}[section]
\newtheorem{theorem}{Theorem}[section]
\newtheorem{corollary}{Corollary}[section]
\newtheorem*{main}{Main Theorem}
\newtheorem{lemma}{Lemma}[section]
\newtheorem{proposition}{Proposition}[section]
\newtheorem{definition}{Definition}[section]
\newtheorem{example}{Example}[section]
\theoremstyle{remark}
\newtheorem*{notation}{Notation}

% Commands and operators
\newcommand{\ind}{\hspace*{0.5cm}}
\newcommand{\gap}{\hspace*{0.25cm}}
\newcommand*{\Cdot}{\raisebox{-0.25ex}{\scalebox{1.3}{$\cdot$}}}
\newcommand{\vect}[1]{\mathbf{#1}}
\newcommand{\transpose}{\text{T}}
\DeclareMathOperator{\interior}{int}
\DeclareMathOperator{\domain}{dom}
\DeclareMathOperator{\diag}{diag}


\begin{document}
\lstset{language=}
\pagestyle{plain}

\title{Notes}
\author{David Wu}

\maketitle

\tableofcontents

\section{Linear regression}\label{s:linear_regression}
    We first recall the standard linear regression model. The set up is as follows:
    \begin{itemize}
        \item $D$ input variables $x_1, \dots, x_D$ with additional dummy input variable $x_0 = 1$. We represent this as an input vector $\vect{x} = (x_0, \dots, x_D)^\transpose$.
        \item $M$ basis functions $\phi_1(\vect{x}), \dots, \phi_{M-1}(\vect{x})$ with additional dummy basis function $\phi_0(\vect{x}) = 1$. We represent this by the vector-valued function $\bm{\phi}(\vect{x}) = (\phi_0(\vect{x}), \dots, \phi_M(\vect{x}))^\transpose$. We may use the notation $\bm{\phi}(\vect{x}) = \bm{\phi}$ also.
        \item $M$ parameters $w_0, \dots, w_{M-1}$ where we $w_0$ is the bias parameter. We represent this by the vector $\vect{w} = (w_0, \dots, w_{M-1})$.
    \end{itemize}
    The model is then given by
    \begin{equation*}
        y(\vect{x}, \vect{w}) = \sum_{j=0}^{M-1} w_j \phi_j(\vect{x}) = \vect{w}^\transpose \bm{\phi}(\vect{x}).
    \end{equation*}
    We now consider finding the optimal parameter vector $\vect{w}$ for our model. Here we have:
    \begin{itemize}
        \item A training set with $N$ data points $\{(\bm{\phi}_n, t_n)\}_{n=1}^{N}$ where $t_n \in \mathbb{R}$ and $\bm{\phi}_n = \bm{\phi}(\vect{x}_n)$. (We've applied $\bm{\phi}$ to the original input data points $\{\vect{x}_n\}_{n=1}^{N}$.) We represent the target data points by the vector $\vect{t} = (t_1, \dots, t_N)^\transpose$.
        \item We introduce the notation $y_n = y(\bm{\phi}_n) = \vect{w}^\transpose \bm{\phi}_n$.
    \end{itemize}
    The optimal parameter vector $\vect{w}$ is the parameter vector $\vect{w}$ which minimizes the \emph{sum-of-squares} error function
    \begin{equation}\label{e:sum_of_squares}
        E(\vect{w}) = \frac{1}{2} \sum_{n=1}^{N} 
                       \{y_n - t_n\}^2.
    \end{equation}
\section{Logistic regression}\label{s:logistic_regression}
    % Useful notation for future use:
    % p(\mathcal{C}_1 | \bm{\phi})
    % p(\mathcal{C}_2 | \bm{\phi}) = 1 -  p(\mathcal{C}_1 | \bm{\phi})
    % -\ln{p(\vect{t} | \vect{w})}
    For logistic regression we start with the setup for linear regression (Section \ref{s:linear_regression}). The logistic regression model is then given by 
    \begin{equation*}
        y(\bm{\phi}) = \sigma(\vect{w}^\transpose \bm{\phi}),
    \end{equation*}
    where $\sigma$ is the logistic sigmoid function defined by
    \begin{equation*}
        \sigma(a) = \frac{1}{1 + \exp(-a)}.
    \end{equation*}
    This function has the useful property
    \begin{equation*}
        \dv{\sigma}{a} = \sigma(1 - \sigma).
    \end{equation*}
    
    We now consider finding the optimal parameter vector $\vect{w}$ for our model. Here we have:
    \begin{itemize}
        \item A training set with $N$ data points $\{(\bm{\phi}_n, t_n)\}_{n=1}^{N}$ where $t_n \in \{0, 1\}$ and $\bm{\phi}_n = \bm{\phi}(\vect{x}_n)$. (We've applied $\bm{\phi}$ to the original input data points $\{\vect{x}_n\}_{n=1}^{N}$.) We represent the target data points by the vector $\vect{t} = (t_1, \dots, t_N)^\transpose$.
        \item We introduce the notation $y_n = y(\bm{\phi}_n) = \sigma(\vect{w}^\transpose \bm{\phi}_n)$.
    \end{itemize}
    The optimal parameter vector $\vect{w}$ is the parameter vector $\vect{w}$ which minimizes the \emph{cross-entropy} error function
    \begin{equation}\label{e:cross_entropy}
        E(\vect{w}) =  - \sum_{n=1}^{N} 
                       \{t_n \ln y_n + (1 - t_n)\ln(1 - y_n)\}.
    \end{equation}
    For solving this minimization problem, the gradient of the error function is useful. It is given by
    \begin{equation*}
        \nabla E(\vect{w}) = \sum_{i=1}^{N} (y_n - t_n)\bm{\phi}_n.
    \end{equation*}

\section{Regularization: linear regression and logistic regression}
    If a model overfits the training data, it may not generalize well to new examples. Here we describe one approach to reduce overfitting. 

    Let us denote both the sum-of-squares error function (Equation \ref{e:sum_of_squares}) and cross-entropy error function (Equation \ref{e:cross_entropy} introduced for linear regression and logistic regression, respectively, by $E_D(\vect{w})$. We refer to $E_D(\vect{w})$ as the \emph{data-dependent} error function. The overall error function we consider is
    \begin{equation*}
        E(\vect{w}) = E_D(\vect{w}) + \lambda E_W(\vect{w}),
    \end{equation*}
    where we call $E_W(\vect{w})$ the regularization term and $\lambda$ the regularization coefficient. 

    There are many choices for $E_W(\vect{w})$. A simple (and common) choice is the sum-of-squares of the elements of the weight vector $\vect{w}$
    \begin{equation*}
        E_W(\vect{w}) = \frac{1}{2}\vect{w}^\transpose \vect{w}.
    \end{equation*}
    The gradient of this quadratic regularizer with respect to $\vect{w}$ is
    \begin{equation*}
        \nabla E_W(\vect{w}) = \vect{w}.
    \end{equation*}

\section{Convex optimization}
    \subsection{Optimization problems}
        In this section we recall some of the basic terminology and definitions used to discuss optimization problems. We use the notation
        \begin{equation}\label{e:opt_problem}
            \begin{aligned}
            & {\text{minimize}} && f_0(x) \\
            & \text{subject to} && f_i(x) \leq 0, \gap i = 1, \dots, m \\
            &                   && h_i(x) = 0, \gap i = 1, \dots, p
            \end{aligned}
        \end{equation}
        to denote the \emph{optimization problem} of finding an $x$ that minimizes $f_0(x)$ among all $x$ that satisfy the conditions $f_i(x) \leq 0, \gap i = 1, \dots, m$ and $h_i(x) = 0, \gap i = 1, \dots, p$. We have the following terminology and definitions:
        \begin{itemize}
            \item $x \in \mathbb{R}^n$ is the \emph{optimization variable}.
            \item $f_0: \mathbb{R}^n \to \mathbb{R}$ is the \emph{objective} or \emph{cost function}.
            \item $f_i(x) \leq 0$ are the \emph{inequality constraints} and $f_i(x)$ are the \emph{inequality constraint functions}.
            \item $h_i(x) = 0$ are the \emph{equality constraints} and $h_i(x)$ are the \emph{equality constraint functions}. 
            \item The \emph{domain} $\mathcal{D}$ of the optimization problem \eqref{e:opt_problem} is the set of points for which the objective and all constraint functions are defined. 
            \item A point $x \in D$ is \emph{feasible} if it satisfies all constraints $f_i(x) \leq 0$ and $h_i(x) = 0$. The set of all feasible points is called the \emph{feasible} or \emph{constraint set}. We say the problem \eqref{e:opt_problem} is \emph{feasible} if a feasible point exists and \emph{infeasible} otherwise.
            \item The \emph{optimal value} of the problem \eqref{e:opt_problem} is
                  $p^\star \coloneqq \inf \{ f_0(x) \;|\; \text{feasible points } x \}$.
            \item The point $x^\star$ is an \emph{optimal point}, that is, solves the problem \eqref{e:opt_problem}, if $x^\star$ is feasible and $f_0(x^\star) = p^\star$. The set of all optimal points is the \emph{optimal set}. 
            \item A point $x$ is \emph{$\epsilon$-suboptimal} if it is feasbile and $f_0(x) \leq p^\star + \epsilon$. The set of all $\epsilon$-suboptimal points is called the \emph{$\epsilon$-suboptimal set}.
        \end{itemize} 

   \subsection{Convex optimization problems}
        In this section we recall some of the basic terminology and definitions used to discuss convex optimization problems. A \emph{convex optimization problem} is an optimization problem of the form 
        \begin{equation}\label{e:convex_opt_problem}
            \begin{aligned}
            & {\text{minimize}} && f_0(x) \\
            & \text{subject to} && f_i(x) \leq 0, \gap i = 1, \dots, m \\
            &                   && a_i^\transpose x = b_i, \gap i = 1, \dots, p
            \end{aligned}
        \end{equation}
        where $f_0, \dots, f_m$ are convex functions. 

        The problems of most interest to us are convex optimization problems of the form  
        \begin{equation}\label{e:convex_opt_problem_of_interest}
            \begin{aligned}
            & {\text{minimize}} && f_0(x) \\
            & \text{subject to} && f_i(x) \leq 0, \gap i = 1, \dots, m
            \end{aligned}
        \end{equation}
        where $f_0, \dots, f_m$ are convex functions. 

    \subsection{Subgradients}
        In this section we recall some of the basics of subgradients. Recall that a vector $g \in \mathbb{R}^n$ is a \emph{subgradient} of $f: \mathbb{R}^n \to \mathbb{R}$ at $x \in \domain f$ if 
        \begin{equation}\label{d:subgrad}
            f(z) \leq f(x) + g^\transpose(z-x)
        \end{equation}
        for all $z \in \domain f$.

        \begin{theorem}[Existence of subgradients]\label{t:existence_of_subgrads}
            If $f$ is convex and $x \in \interior \domain f$, then the subdifferential $\partial f(x)$ is nonempty and bounded 
        \end{theorem}

        \begin{theorem}[Subgradients of differentiable functions]\label{t:subgrads_of_diff_functions}
            If $f$ is convex and differentiable at $x$, then $\partial f(x) = \{ \nabla f(x)\}$.
        \end{theorem}

        \begin{theorem}[Minimums and subgradients]\label{t:minimums_and_subgrads}
            A point $x^\star$ is a minimizer of a function $f$ if and only if $f$ is subdifferentiable at $x^\star$ and $0 \in \partial f(x^\star)$.
        \end{theorem}

\section{Algorithms}
    \subsection{The set-up}
        In this section we discuss methods for solving the unconstrained optimization problem 
        \begin{equation*}\label{e:unconstrained_opt_problem}
            \min_{\domain f} f(x)
        \end{equation*}
        where 
        \begin{itemize}
            \item $f: \mathbb{R}^n \to \mathbb{R}$ is convex and twice continuously differentiable.
            \item We assume the problem is solvable, i.e., there exists an optimal point $x^\star$, and denote the optimal value by $p^\star \coloneqq f(x^\star)$.
        \end{itemize} 
        Now, as $f$ is differentiable and convex, a point $x^\star$ is optimal if and only if $\nabla f(x^\star) = 0$, and so the problem \eqref{e:unconstrained_opt_problem} is equivalent to finding a solution to $\nabla f(x) = 0$, a set of $n$ equations in $n$ variables. In general, solving $\nabla f(x) = 0$ analytically is impossible so we aim to find an approximate solution via a iterative algorithm. That is, an algorithm that computes a sequence of points $x^{(0)}, x^{(1)}, \dots \in \domain f$ where $f(x^{(k)}) \to p^\star$ as $k \to \infty$, called  a \emph{minimizing sequence} for the problem \eqref{e:unconstrained_opt_problem}, and terminates when $f(x^{(k)}) - p^\star \leq \epsilon$, where $\epsilon > 0$ is a given \emph{tolerance}. 
    \subsection{Descent methods}
        In this section we introduce the \emph{conceptual} (or \emph{general}) \emph{descent method} which we can see as the template on which many common minimization algorithms are based. It is as follows:

        \begin{algorithm}[Conceptual descent method]
        \label{a:conceptual_descent_method}\mbox{}\\
            \ind \textbf{given} a starting point $x \in \domain f$\\
            \ind \textbf{repeat} \\
            \ind\ind 1. Determine a descent direction $\Delta x$. \\
            \ind\ind 2. \emph{Line search.} Choose a step size $t > 0$. \\
            \ind\ind 3. \emph{Update.} $x \coloneqq x + t \Delta x$. \\
            \ind \textbf{until} stopping criterion is satisfied. \\
        \end{algorithm} 
    \subsection{Backtracking line search}
         \begin{algorithm}[Backtracking line search]
        \label{a:basic_conceptual_cp_alg}\mbox{}\\
            \ind \textbf{given} a descent direction $\Delta x$ for $f$ at $x \in \domain f$, $\alpha \in (0, 0.5), \beta \in (0, 1).$ \\
            \ind $t \coloneqq 1$. \\
            \ind \textbf{while} $f(x + t\Delta x) > f(x) + \alpha t \Delta f(x)^\transpose \Delta x$ \\
            \ind\ind $t \coloneqq \beta t$.
        \end{algorithm}        

    \subsection{Newton's method}
        Here we discuss \emph{Newton's method}.

        \begin{algorithm}[Newton's method]
        \label{a:basic_conceptual_cp_alg}\mbox{}\\
            \ind \textbf{given} a starting point $x \in \domain f$, tolerance $\epsilon > 0.$ \\
            \ind \textbf{repeat} \\
            \ind\ind 1. \emph{Compute the Newton step and decrement.} \\
            \ind\ind\ind $\Delta x_{nt} \coloneqq -\nabla^2 f(x)^{-1} \nabla f(x)$ \\
            \ind\ind\ind $\lambda^2 \coloneqq \nabla f(x)^\transpose \nabla^2 f(x)^{-1} \nabla f(x).$ \\
            \ind\ind 2. \emph{Stopping criterion.} \textbf{quit} if $\lambda^2/2 \leq \epsilon.$\\
            \ind\ind 3. \emph{Line search.} Choose step size $t$ by backtracking line search. \\
            \ind\ind 4. \emph{Update.} $x \coloneqq x + t\Delta x_{nt}.$ \\
        \end{algorithm} 

\section{Cutting plane methods}
    \subsection{The basic conceptual cutting-plane/localization algorithm}
        In this section we describe and present the basic conceptual cutting-plane (or localization) algorithm. The initial set-up is as follows:
        \begin{itemize}
            \item A target set $X$.
            \item A set of linear inequalities $Cz \preceq d$ with $C \in \mathbb{R}^{q \times n}$ which is satisfied by all $z \in X$. Equivalently, we have a polyhedron $\mathcal{P}_0 \coloneqq \{ z \:|\: Cz \preceq d\}$ with $X \subseteq \mathcal{P}_0$. 
        \end{itemize}
        The purpose of this algorithm is to find a point in a target set $X$ or determine $X$ is empty starting with the knowledge that $X$ is contained within a polyhedron $\mathcal{P}_0$. The algorithm does this by, at each iteration, cutting away an additional section of the polyhedron $\mathcal{P}_0$ until either a point in $X$ is found or nothing is left of $\mathcal{P}_0$, that is, $\mathcal{P}_0 = \emptyset$. The case of most interest to us is when the target set $X$ is the optimal set for the inequality constrained convex optimization problem \eqref{e:convex_opt_problem_of_interest} and $\mathcal{P}_0$ is some polyhedron that contains $X$.

        We begin by choosing a point $x^{(1)} \in \mathcal{P}_0$. We then query the oracle at $x^{(1)}$. If we have that $x^{(1)} \in X$ we quit. Otherwise, $x^{(1)} \not\in X$, so we find a cutting plane 
        \begin{equation*}
            a_1^{\transpose}z \leq b_1
        \end{equation*}
        where
        \begin{equation*}
            a_1^{\transpose}x^{(1)} \geq b_1.
        \end{equation*}
        That is, $x^{(1)}$ sits outside or on the boundary of this cutting plane. We then put 
        \begin{equation*}
            \mathcal{P}_1 \coloneqq \mathcal{P}_0 \cap \{z \:|\: a_1^{\transpose}z \leq b_1\}.
        \end{equation*}
        If $\mathcal{P}_1 = \emptyset$, we quit. Otherwise, $\mathcal{P}_1 \neq \emptyset$, and we repeat this procedure again by choosing a point $x^{(2)} \in \mathcal{P}_1$. In summary:
        \begin{algorithm}[Basic conceptual cutting-plane (or localization) algorithm]
        \label{a:basic_conceptual_cp_alg}\mbox{}\\
            \ind \textbf{given} an initial polyhedron $\mathcal{P}_0 = \{z \:|\: Cz \preceq d\}$ known to contain $X$. \\
            \ind $k \coloneqq 0$. \\
            \ind \textbf{repeat} \\
            \ind\ind Choose a point $x^{k+1} \in \mathcal{P}_k$. \\
            \ind\ind Query the cutting-plane oracle at $x^{(k+1)}$. \\
            \ind\ind If the oracle determines that $x^{(k+1)} \in X$, quit. \\
            \ind\ind Else, update $\mathcal{P}_k$ by adding the new cutting-plane: $\mathcal{P}_{k+1} \coloneqq \mathcal{P}_k \cap \{z \:|\: a_{k+1}^\transpose z \leq b_{k+1} \}$. \\
            \ind\ind If $\mathcal{P}_{k+1} = \emptyset$, quit. \\
            \ind\ind $k \coloneqq k+1$.
        \end{algorithm}
        As noted, this algorithm is conceptual, it requires, in particular, specification of the following: 
        \begin{itemize}
            \item How to construct an appropriate cutting plane if it is determined that the query point $x^{(k+1)}$ for a given iteration $k$ is not in the target set $X$.
            \item How to generate the query point $x^{(k+1)}$ for each iteration $k$.
        \end{itemize}

    \subsection{Cutting planes for inequality constrained problems}
        In this section we describe how to construct cutting planes for the conceptual cutting plane algorithm (Algorithm \eqref{a:basic_conceptual_cp_alg}) when the target set $X$ is the optimal set for the inequality constrained convex optimization problem \eqref{e:convex_opt_problem_of_interest} and $P_0$ is some polyhedron that contains $X$.

        Suppose we have generated a query point $x$. We first check if $x$ is a feasible point. Suppose $x$ is not a feasible point. Then there exists some index $j \in \{1, \dots, m\}$ with $f_j(x) > 0$, that is, the $j$-th constraint has been violated. The cutting plane we will define comes from the following discussion. As $f_j$ is convex (we make the additional assumption $x \in \interior \domain f_j$) there exists some subgradient $g_j \in \partial f(x)$ (Theorem \ref{t:existence_of_subgrads}) with the defining property
        \begin{equation*}
            f_j(z) \geq f_j(x) + g_j^\transpose(z-x)
        \end{equation*}
        for all $z \in \domain f_j$. Therefore,
        \begin{equation*}
            f_j(x) + g^\transpose(z-x) > 0 \implies f_j(z) > 0, \text{ or equivalenty, }
            f_j(z) \leq 0 \implies f_j(x) + g^\transpose(z-x) \leq 0
        \end{equation*}
        for all $z \in \domain f$. That is all feasible points must satisfies $f_j(x) + g^\transpose(z-x) \leq 0$. So the cutting plane is defined to be all points $z$ that satisfy
        \begin{equation*}
            f_j(x) + g^\transpose(z-x) \leq 0.
        \end{equation*}
        Observe that substituting $x$ into this inequality reduces to $f_j(x) \leq 0$ but in actuality $f_j(x) > 0$, so therefore $x$ is not included in this cutting plane, as desired. 

        On the other hand, suppose $x$ is a feasible point. We choose any subgradient $g_0 \in \partial f_0(x)$. If $g_0 = 0$, then $x$ is optimal and we are done (Theorem \ref{t:minimums_and_subgrads}). So suppose $g_0 \neq 0$. Observe that if a point $z$ satisfies $g_0^\transpose(z-x) > 0$, then this implies $f_0(z) \geq f_0(x)$ by the defining inequality (Equation \eqref{d:subgrad}) of subgradients. That is, $z$ is not optimal. So in this case the cutting plane is defined to be all points $z$ that satisfy
        \begin{equation*}
            g_0^\transpose(z-x) \leq 0.
        \end{equation*}

\section{The analytic center cutting-plane method}
    \subsection{The algorithm}
        The analytic center cutting-plane method (ACCPM) is a cutting plane algorithm with both theoretical and practical applicability. For our purposes we specialise to convex optimization problems of the form 
        \begin{equation*}
            \begin{aligned}
            & {\text{minimize}} && f_0(x) \\
            & \text{subject to} && f_i(x) \leq 0, \gap i = 1, \dots, m
            \end{aligned}
        \end{equation*}
        where $f_0, \dots, f_m$ are convex functions. We introduce the ACCPM here, and will discuss details of the algorithm and its implementation in seperate sections, namely:
        \begin{itemize}
            \item How to construct an appropriate cutting plane if it is determined that the query point $x^{(k+1)}$ for a given iteration $k$ is not in the target set $X$.
            \item How to generate the query point $x^{(k+1)}$ for each iteration $k$.
        \end{itemize}
        For convenience of the reader we will review some of what has been discussed in previous sections. The ACCPM is as follows: 
        \begin{algorithm}[Analytic center cutting-plane method (ACCPM)]
        \label{a:basic_conceptual_cp_alg}\mbox{}\\
            \ind \textbf{given} an initial polyhedron $\mathcal{P}_0 = \{z \:|\: Cz \preceq d\}$ known to contain $X$. \\
            \ind $k \coloneqq 0$. \\
            \ind \textbf{repeat} \\
            \ind\ind Compute $x^{(k+1)}$, the analytic center of $\mathcal{P}_k$. \\
            \ind\ind Query the cutting-plane oracle at $x^{(k+1)}$. \\
            \ind\ind If the oracle determines that $x^{(k+1)} \in X$, quit. \\
            \ind\ind Else, update $\mathcal{P}_k$ by adding the new cutting-plane: $\mathcal{P}_{k+1} \coloneqq \mathcal{P}_k \cap \{z \:|\: a_{k+1}^\transpose z \leq b_{k+1} \}$. \\
            \ind\ind If $\mathcal{P}_{k+1} = \emptyset$, quit. \\
            \ind\ind $k \coloneqq k+1$.
        \end{algorithm}
        In the algorithm $X$ is either the optimal set or $\epsilon$-suboptimal set, for some fixed $\epsilon > 0$, for the optimization problem.

    \subsection{Query point generation for the ACCPM: the analytic center}
        At each iteration $k$ the query point $x^{(k+1)}$ is given by the \emph{analytic center} of the inequalities that specify $\mathcal{P}_k$, which we discuss in this section. Consider the function,
        \begin{equation}
            \phi(x) = - \sum_{i=1}^{m+k}{\log{(b_i - a_i^\transpose x)}}
        \end{equation}
        with 
        \begin{equation}
            \domain \phi = \{x \;|\; a_i^\transpose x < b_i, i = 1, \dots, m, m+1, \dots, m+k\}
        \end{equation}
        called the \emph{logarithmic barrier} or \emph{log barrier} for (the inequalities which specify) the polyhedron $\mathcal{P}_k$. (We may simplify the constraint inequalities as $Ax \preceq b$.) The analytic center of the inequalities which specify the polyhedron $\mathcal{P}_k$ is then given by the solution of the problem 
        \begin{equation}\label{e:log_barrier_problem}
            \min_{\domain \phi} \phi.
        \end{equation}
        We will refer to this simply as the analytic center of $\mathcal{P}_k$ for convenience while acknowledging that $\mathcal{P}_k$ can be specified by (infinitely) many different choices of half planes and that this choice determines the solution of \eqref{e:log_barrier_problem}.

    \subsection{Computing the analytic center}
        The approach we take in computing the analytic center is as follows:
        \begin{enumerate}
            \item Use a phase I optimization method to find a point in $\domain \phi$.
            \item Use the standard Newton method to compute the analytic center.
        \end{enumerate}
        We first discuss the Newton method in the context of computing the analytic center. We give the Newton method applied to $\phi$
        \begin{algorithm}[Newton's method]
        \label{a:basic_conceptual_cp_alg}\mbox{}\\
            \ind \textbf{given} a starting point $x \in \domain \phi$, tolerance $\epsilon > 0.$ \\
            \ind \textbf{repeat} \\
            \ind\ind 1. \emph{Compute the Newton step and decrement.} \\
            \ind\ind\ind $\Delta x_{nt} \coloneqq -\nabla^2 \phi(x)^{-1} \nabla \phi(x)$ \\
            \ind\ind\ind $\lambda^2 \coloneqq \nabla \phi(x)^\transpose \nabla^2 \phi(x)^{-1} \nabla \phi(x).$ \\
            \ind\ind 2. \emph{Stopping criterion.} \textbf{quit} if $\lambda^2/2 \leq \epsilon.$\\
            \ind\ind 3. \emph{Line search.} Choose step size $t$ by backtracking line search. \\
            \ind\ind 4. \emph{Update.} $x \coloneqq x + t\Delta x_{nt}.$ \\
        \end{algorithm} 
        For the gradient and Hessian of $\phi$ we have
        \begin{align}
            &\nabla \phi(x) = \sum_{i=1}^{m+k} \frac{1}{b_i - a_i^\transpose x}a_i = A^\transpose d \\
            &\nabla^2 \phi(x) = \sum_{i=1}^{m+k} \frac{1}{(b_i - a_i^\transpose x)^2}a_i a_i^\transpose = A^\transpose \diag(d)^2 A
        \end{align}
        where for $d \in \mathbb{R}^{m+k}$ we have
        \begin{align*}
            d_i = \frac{1}{b_i - a_i^\transpose x}.
        \end{align*}
        Observe that since $x$ is strictly feasible, $d \succ 0$, which implies $\det(d) \neq 0$ 





\renewcommand\refname{Bibliography}
\begin{thebibliography}{99}
    \bibitem[Bis06]{bishop_06} C. Bishop. \emph{Pattern Recognition and Machine Learning}. Springer, 2006.
    \bibitem[BV04]{boyd_vandenberghe_04} S. Boyd and L. Vandenberghe. \emph{Convex Optimization}. Cambridge University Press, 2004.
    \bibitem[BVS08]{boyd_vandenberghe_skaf_08} S. Boyd, L. Vandenberghe and J. Skaf. \emph{Analytic Center Cutting-Plane Method}. 2008. \url{https://see.stanford.edu/materials/lsocoee364b/06-accpm_notes.pdf}.
    \bibitem[BV11]{boyd_vandenberghe_11} S. Boyd and L. Vandenberghe. \emph{Localization and Cutting-Plane Methods}. 2011. \url{http://web.stanford.edu/class/ee364b/lectures/localization_methods_notes.pdf}.
    \bibitem[BDV15]{boyd_duchi_vandenberghe_11} S. Boyd, J. Duchi and L. Vandenberghe. \emph{Subgradients}. 2015. \url{http://web.stanford.edu/class/ee364b/lectures/subgradients_notes.pdf}.
    \bibitem[NgA]{ng_a} A. Ng. \emph{CS229 Lecture Notes 1}. \url{http://cs229.stanford.edu/notes/cs229-notes1.pdf}.
    \bibitem[NgB]{ng_b} A. Ng. \emph{Machine Learning}. \url{https://www.coursera.org/learn/machine-learning}.
    \bibitem[ShaA]{shalizi_a} C. R. Shalizi. \emph{Advanced Data Analysis
    from an Elementary Point of View}. \url{http://www.stat.cmu.edu/~cshalizi/ADAfaEPoV/ADAfaEPoV.pdf}. 
\end{thebibliography}

% Examples:
% \bibitem[AHU]{ahu} Aho, A.,\ Hopcroft, J.,\ and Ullman, J.\ (1976). {\em{The Design and Analysis of Computer Algorithms.}} Addison Wesley, Reading, Mass.

% \bibitem[AT]{AT} Auslander, L. and Tolmieri, R. (1979). Is Computing with the Fast Fourier Transform Pure or Applied Mathematics? {\em{Bulletin (New Series) of the AMS Vol. 1}} 847-897.

% [Lee03] John Lee, \emph{Introduction to Topological Manifolds}, Springer Science, New York, USA, 2003.

% [Rat06] John Ratcliffe, \emph{Foundations of Hyperbolic Manifolds}, Springer Science, New York, USA, 2006.

\end{document}